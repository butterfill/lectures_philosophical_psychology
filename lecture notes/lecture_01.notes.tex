 %!TEX TS-program = xelatex
%!TEX encoding = UTF-8 Unicode

%\def \papersize {a5paper}
\def \papersize {a4paper}
%\def \papersize {letterpaper}

%\documentclass[14pt,\papersize]{extarticle}
\documentclass[12pt,\papersize]{extarticle}
% extarticle is like article but can handle 8pt, 9pt, 10pt, 11pt, 12pt, 14pt, 17pt, and 20pt text

\def \ititle {Origins of Mind: Lecture Notes}
\def \isubtitle {Lecture 01}
%comment some of the following out depending on whether anonymous
\def \iauthor {Stephen A.\ Butterfill}
\def \iemail{s.butterfill@warwick.ac.uk% \& corrado.sinigaglia@unimi.it
}
%\def \iauthor {}
%\def \iemail{}
%\date{}

%\input{$HOME/Documents/submissions/preamble_steve_paper4}
\input{$HOME/Documents/submissions/preamble_steve_lecture_notes}

%no indent, space between paragraphs
\usepackage{parskip}

%comment these out if not anonymous:
%\author{}
%\date{}

%for e reader version: small margins
% (remove all for paper!)
%\geometry{headsep=2em} %keep running header away from text
%\geometry{footskip=1.5cm} %keep page numbers away from text
%\geometry{top=1cm} %increase to 3.5 if use header
%\geometry{bottom=2cm} %increase to 3.5 if use header
%\geometry{left=1cm} %increase to 3.5 if use header
%\geometry{right=1cm} %increase to 3.5 if use header

% disables chapter, section and subsection numbering
\setcounter{secnumdepth}{-1}

%avoid overhang
\tolerance=5000

%\setromanfont[Mapping=tex-text]{Sabon LT Std}


%for putting citations into main text (for reading):
% use bibentry command
% nb this doesn’t work with mynewapa style; use apalike for \bibliographystyle
% nb2: use \nobibliography to introduce the readings
\usepackage{bibentry}

%screws up word count for some reason:
%\bibliographystyle{$HOME/Documents/submissions/mynewapa}
\bibliographystyle{apalike}


\begin{document}



\setlength\footnotesep{1em}






%---------------
%--- start paste
\title {Philosophical Psychology \\ Lecture 01: Seeing Red: Do Humans Visually Experience Categorical Colour Properties?}



\maketitle

\subsection{title-slide}
This is a module on Puzzling Questions about Minds in Action.
I want to consider a range of cases in which scientific findings
create the sort of puzzles that philosophers seem well-placed
to grapple with.

My plan is to cover a range of topics, as listed on the course outline.
There are several themes that are more or less independent of each other.
I wanted to start with a question about colour experiences because
partly this is a fun introduction.
The other lectures aren’t closely related to this one: I’m guessing that
some people won’t realise the module starts in week 1.



\section{A ‘subject-determining platitude’ about colour}

According to Frank Jackson (1996, pp. 199–200),
it is a ‘subject-determining platitude’
that ‘“red” denotes the property of an object putatively presented in visual experience
when that object looks red’, and likewise for other colour terms.

\subsection{slide-3}
Here are three patches of colour.
The patches are all different colours, but the two leftmost are both the
same colour---they are both blue.
This sounds contradictory but isn't.
In one case we're talking about the particular colours of things,
which I’ll call ‘shades’; in the
other case we're talking about colour category.

Today my focus is not the shades but the categorical colour properties,
properties like red, green and blue.

\subsection{slide-4}
It has been quite widely accepted (among philosophers, at least)  that:
%
\begin{quote}
‘If someone with normal color vision looks at a tomato in good light, the tomato will appear to have a distinctive property—a property that strawberries and cherries also appear to have, and which we call ‘red’ in English’ \citep[p.\ 4]{byrne:2003_color}
\end{quote}
%
Claims such as this are sometimes treated as common ground in controversies about colour.

\subsection{slide-6}
Is this true?
Does ‘“red” denote the property of an object putatively presented in visual experience
when that object looks red’?

\subsection{slide-7}


\section{Do you visually experience red because you call things ‘red’?}

“surprising it would be indeed if I have a perceptual experience as of red
because I call the perceived object ‘red’”
(Stokes 2006, pp. 324--5).

\subsection{slide-11}
“surprising it would be indeed if I have a perceptual experience as of red because I call the perceived object ‘red’.”
\citep[pp.~324--5]{Stokes:2006fd}.

Stokes makes this observation in passing.  I want to show
that what Stokes finds so suprising is true, or would be if it were true that
red things differ in visual appearance from non-red things.

\subsection{slide-12}
Stokes has formulated things badly here.
The important thing isn’t the particular word I use, ‘red’ vs, say, ‘rot’,
‘rosso’ or ‘rose’.
Rather it’s that I have a label for the perceived object which I also
use for all the things that have the property red.

\subsection{slide-15}
It turns out that
the ability to discriminate properties denoted by particular colour terms like
‘red’ depends not only on having learned to use those very terms accurately in
the past \citep{Ozgen:2002yk,Winawer:2007im,zhou:2010_newly} but also on being
able to activate some component of the ability to apply the colour term at the
time a stimulus is presented
\citep{Roberson:2000ge,Pilling:2003bi,Wiggett:2008xt}. % page refs: (Roberson,
Davies and Davidoff 2000: 985; Pilling, Wiggett, et al. 2003: 549-50; Wiggett
and Davies 2008)

Someone who accepts that there are visual experiences as of
\emph{red} must either suppose that these experiences are only indirectly
related to abilities to discriminate or else accept the surprising idea that
such visual experiences are a consequence of covert labelling. This dilemma can
be avoided by rejecting the subject-determining platitude and with it the
existence in humans of visual experiences as of \emph{red}.

\subsection{slide-17}
Should we reject this ...

\subsection{slide-18}
... or this (or both)?

\subsection{slide-13}


\section{How to Measure Phenomenology}

It is perhaps tempting to assume that claims about phenomenology
cannot be tested scientifically.
But this is a mistake.  We can use experiments to address the question,
Do things which have the property denoted by ‘red’ thereby appear to be different from things which lack it?

\subsection{slide-19}
Recall that our question is,
Do red things differ in visual appearance from non-red things?

\subsection{slide-20}
How could we tell whether
things which have the property denoted by ‘red’ thereby appear to be different from things which lack it?

\subsection{slide-21}
An immediate problem arises from the fact that pairs of things  just one of which is \emph{red} (i.e. has the property denoted by ‘red’) differ not only in this way but also in which particular shade of colour they have.
How can we tell whether differences in visual appearance are due in part to differences in being \emph{red} rather than entirely due to differences in shade?

\subsection{slide-22}
The problem can be overcome by introducing a third thing.  Consider constructing a sequence of three things which are indiscriminable except by colour.  Let the middle thing be \emph{red}, and let the first thing not be \emph{red}.  Now consider the difference between the particular hues of these two things.  Ensure the difference is small, but large enough that any two things which differ in hue by this amount are readily discriminable.
Finally, let the third thing be \emph{red}, and let the difference with respect to hue between the third and second thing be the same as the difference between the first and second thing.  (Research on colour perception shows that such differences can be equated, and that sequences of this type exist; see
\citealp{kuehni:2001_color} on perceptual uniformity and \citealp{witzel:2013_categorical} on categories.)
Consider the visual appearances of these three things.

\subsection{slide-24}
If, as the subject-determining platitude implies, humans visually experience properties like \emph{red}, then (given the premise) the first and second thing should differ in visual appearance more than the second and third thing.

\subsection{slide-25}
How can we test whether such differences in visual appearance  exist?
Several methods have recently been used \citep{witzel2014category,webster:2012_color,davidoff:2012_perceptual}. %(*cite thesis?).

\subsection{slide-26}
In one people are asked to judge, for each sequence, which of the two outer things (i.e. the first and third thing) is more similar to the middle thing (i.e. the second thing).
Given that visual appearances typically influence judgements of similarity,
if things which differ in whether they are \emph{red} thereby differ in visual appearance, we would expect people to judge that the outer thing which is
\emph{red} is more similar to the middle thing than the other outer thing.

\subsection{slide-27}
Another method involves asking people to adjust the colour of the middle object so that it appears to be mid-way between the two outer objects.
What people  are in fact adjusting here is the hue of the object, but no mention is made of hue: their instructions are to match differences in appearance.
If things which differ in whether they are \emph{red} thereby differ in visual appearance, we would expect people to compensate for this in adjusting hue.
In fact they do not \citep{witzel2014category}.

\subsection{slide-28}
I’ll explain this method in a while.
First,

\subsection{slide-29}
Let’s consider the first method, which involves
making judgements about similarity.

\subsection{slide-30}
In brief: Kay and Kempton contrasted the responses of native English
speakers (who have words for green and blue) with native Tarahumara (a Uto-Aztecan language of
northern Mexico) speakers, whose basic colour words mark no such distinction.

Here you see a triad of colours, A, B and C.
Kay and Kempton first measured ‘discrimination distance’.
That is, how far apart are each of these in terms of JNDs?
As it turns out, JNDs are not affected by which categorical colour properties you can name
\citep{witzel:2013_categorical}.
So we would expect discrimination distance to be approximately the same for all
participants. (Some studies do measure discrimination distance for
each subject individually and find indiviudal differences; e.g. \citep{witzel:2014_categorical}).
In this case, you can see that A is further from B in JNDs than B is from C.

‘discrimination distance’:
‘The scale of psychological distance between colors we take as the "real" scale
for present pur- poses is called discrimination distance. The unit of this
scale is the just noticeable dif- ference (jnd), that is, the smallest physical
difference in wavelength that can be detected by the human eye.’
\citep[p.~68]{kay_what_1984}

Next Kay and Kempton measured how visually similar their subjects judged these
samples to be.
To do this, they showed them different triads of colours and asked them,
Which is the most different from the other two?
For each pair they then computed the proportion of times that pair was split.
(As they write: ‘The psychological distance between A and B relative to other
stimulus pairs in the set is given by the proportion of times A and B are split
by the subject's selection of one of them as the most different item in the
triad.’ p. 70.)
This is what you see from the Tarahumara speakers and the English speakers
under the circles.

Looking at the numbers, you can see that
the English speakers tended to split B and C more often than A and B,
whereas the Tarahumara speakers did the opposite.

And note that the B-C pair crosses the blue-green boundary.
What does this mean?
‘The presence of the blue-green lexical category boundary appears to cause
speakers of English to exaggerate the subjective distances of colors close to
this boundary. Tarahumara, which does not lexicalize the blue-green contrast,
does not show this distorting effect.’

\citep[p.~77]{kay_what_1984}:
‘the English speaker judges chip B to be more similar to A than to C because
the blue-green boundary passes between B and C, even though B is perceptually
closer to C than to A.’

This is exactly the sort of evidence that should persuade us
that red things differ in visual appearance from non-red things.

\subsection{slide-31}
Here’s another comparison.
In this case, the discrimination distance (in JNDs) was the same between
the three colour samples.
Both groups thought that B and C are most different, and these cross
the blue-green boundary.  But this effect was stronger in the English speakers.

\subsection{slide-32}
There were some other comparisons that I won’t talk about.

Overall, this is evidence for the conclusion that
red things differ in visual appearance from non-red things.

\subsection{slide-33}
Is the effect due to

visual appearances

or merely to

the ‘Name Strategy’?


The ‘name strategy’: ‘We propose that faced with this situation the
English-speaking subject reasons
unconsciously as follows: “It's hard to decide here which one looks the most
different. Are there any other kinds of clues I might use? Aha! A and B are
both CALLED green while C is CALLED blue. That solves my problem; I'll pick C
as most different.” ... this cognitive strategy ... we will call the
“name strategy”’ \citep[p.~72]{kay_what_1984}.

‘According to the name strategy hypothesis, the speaker who is
confronted with a difficult task of classificatory judgment may use the lexical
classification of the judged objects as if it were correlated with the required
dimension of judgment even when it is not, so long as the structure of the task
does not block this possibility’ (p. 75).

\subsection{slide-34}
They consider a modification to block
use of the naming strategy.
This involves showing subjects only two of the three
stimuli at any one time (Experiment 2).

(‘The three chips were arranged in a
container with a sliding top that permitted the subject to see alternately
either of two pairs of the three chips, but never all three at once. For
example, in triad (A, B, C) the pairs alternately made visible were (A, B) and
(B, C).’) When they do this, categorical colour properties have no effects on
perceptual judgements of similarlity. ‘Subjective similarity judgments follow
discrimination distance and reflect no influence from lexical category
boundaries’ (p. 73).

\subsection{slide-35}
Here are the results from experiment 2.
In this case, there are only English speaking subjects.
And the numbers for the subjects are
‘simply the number out of 21 subjects who chose the indicated pairwise subjective distance as larger.’

The results now are quite different.
Top left: subjects appear to go with discriminability (JND distance)
rather than colour category.
Bottom left: when discriminability (JND distance) is equated,
subjects show no significant effect of colour category (although there is a trend).

‘Subjective similarity judgments follow
discrimination distance and reflect no influence from lexical category
boundaries.’ \citep[p.~73]{kay_what_1984}

\subsection{slide-36}
Conclusion: the name strategy explains the effects of Experiment 1.
Red things do not differ in visual appearance from non-red things

\subsection{slide-37}
They changed the instructions between Experiment 1 and Experiment 2.

\subsection{slide-38}
This change invites the objection that the instruction
encouraged
subjects to attend to hue and ignore other features of the colour.
So the results are inconclusive.

\subsection{slide-39}
Why spend so long on flawed research from which we can’t draw a conclusion
either way?  To illustrate that the issue is quite complex,
and that you have to read papers carefully!
(You aren’t supposed to discuss minutae of papers in your essays;
you’re not an expert on stats or methods.  But if you cite a paper
in support of a claim, you’d better be as sure as you can that the
paper really does support that claim.)

\subsection{slide-40}
According to more recent research discussed in \citep{witzel2014category}
(a published report is not yet available), in fact they do not. That is,
whether things are \emph{red} appears to make no difference to judgements
of similarity (\citealp{witzel2014category}; actually these researchers did
not test ‘red’, but they did test the German terms ‘Rosa’, ‘Braun’,
‘Orange’, ‘Gelb’, ‘Gruen’, ‘Blau’, and ‘Lila’.) This indicates that things
which are \emph{red} do not thereby differ in visual appearance from things
that are not \emph{red}.

\subsection{slide-41}
From correspondence: ‘that experiment was originally reported in the same
paper as the "Categorical facilitation" one, but it was too long. Up to now
it is only in my thesis and there is the abstract, which I paste you below.
Note, however, that we found slight traces of perceptual magnet effects
around prototypes when reanalysing the data later on, which is reflected in
the book chapter, but not the older abstract. The found effects are quite
systematic (happening at almost all prototypes), but also extremely weak,
and there were literally no effects at boundaries.’

How can we defend the view that red things differ in visual appearance from
non-red things? Maybe there are differences in visual appearance that we
are not aware of? Maybe the judgements of similarlity are not sensitive
enough? (But note that \citet{witzel2014category} did find evidence for the
effect of prototypes on judgements of similarity, so the method seems good:
to test this, the three colour samples were arranged so that the prototype
was positioned between a middle and an outer sample. ‘Colours close to the
prototypes were judged to be more similar than they actually were in terms
of pure discrimination’.)

\subsection{slide-44}
Another method involves asking people to adjust the colour of the middle object so that it appears to be mid-way between the two outer objects.
What people  are in fact adjusting here is the hue of the object, but no mention is made of hue: their instructions are to match differences in appearance.
If things which differ in whether they are \emph{red} thereby differ in visual appearance, we would expect people to compensate for this in adjusting hue.
In fact they do not \citep{witzel2014category}.
%
%*NB: second approach (and third approach = perceptual grouping) matters
because you might say that the things do appear differently, just in a
way that we are unaware of. Unawareness of the difference less plausibly
affects grouping and hue matching; after all, if it doesn't affect these
what does the difference in appearance affect?

How can we defend the view that red things differ in visual appearance from non-red things?
You could insist that the difference in appearance is not manifest in this particular case.
Perhaps, for example, because the slider people used to ajust the middle sample changed
its hue, subjects attended to hue and ignored other features of the colour.
(There’s no evidence for this speculation and subjects were not asked about the hue.)
Against this possibility, consider that \citep{witzel2014category} report that they
did find effects for prototypes.
(p. 207: ‘According to the idea of a perceptual magnet effect (e.g. Kuhl 1991), it is also possible that similarity is perceived to be greater around the prototype. In this case, the presence of a prototype between a triad extreme and the discrimination centre should reduce their apparent di erence, and the point of subjective centrality should be shi ed away from the prototype.’)
(p. 207: ‘both tasks showed prototype e ects in centre triads. Colours close to the prototypes were judged to be more similar than they actually were in terms of pure discrimination.  is  finnding indicates that colour differences subjectively appear to be lower around the prototype.’)
The fact that prototypes had an effect indicates that subjects did not focus exclusively
on hue.
This makes it harder to understand why, if categorical colour properties influence
visual appearance, the categorical colour properties did not influence visual
judgements about which of the three colour samples differed most from the others.

Are there other objections?

Could you object that the difference in appearance between red and non-red things
is already built into the metric used for the colour space?  In that case you would not
expect additional effects of categorical colour property on appearance. These effects would be
already taken into account in the measurements of similarity.

To answer this objection we need to note that there are multiple ways of
measuring perceptual similarity.
In this case, the researchers used JNDs.
They have previously shown that category boundaries (as specified by an individual
subject’s colour words) do not affect JNDs.  That is, JNDs are not generally smaller at the
category boundaries \citep{witzel:2013_categorical}.
But when you take pairs of colours that are, say, 3 JNDs apart, then whether or not
the pair straddles a category boundary will affect speed and accuracy of discimination
\citep{witzel:2014_categorical}.
So we can be confident that using JNDs to measure perceptual similarity does not
take into account differences in the ways categorical colour propreties appear.

\subsection{slide-46}
Converging evidence that things labelled with different basic colour terms
do not thereby differ in visual appearance involves a third method for
detecting visual appearances (\citealp{webster:2012_color}). The idea, as
illustrated in Figure \vref{fig:perceptual_grouping}, is that how things
appear with respect to colour should influence how likely it is that they
will be grouped together perceptually. The results are consistent with the
view that whether objects are labelled with the same basic colour term
makes no measurable difference to how they are grouped perceptually. This
further supports suspicion that properties denoted by colour terms like
‘red’ make no difference to visual appearances (see also
\citealp{davidoff:2012_perceptual}).

Let’s consider the experiment in more detail.
‘circles at opposite diagonal corners of the square had the same color,
while the two diagonals differed in color by a fixed angle of 30°’.

The center circle could either be the same colour as one of the diagonals
(as in the left and right panels),
or it could be an inbetween colour (as in the middle panel).
When the central colour is the same as one of the diagonals, you should perceptually
group the diagonal as a line, like \/ or \\.

Subjects’ task was indeed to judge which way the diagonal went.
\citet{webster:2012_color} varied the center colour to find at what point
a given subject was equally like to judge that the line was \/ or \\.
Call this the ‘grouping midpoint colour’.
(p. 378: ‘Observers made a two-alternative forced choice response to indicate
whether the perceived orientation was clockwise or counterclockwise. A
staircase varied the center color angle to estimate the an- gle at which
both orientations appeared equally likely, with the point of subjective
equality estimated from the mean of the final 10 of 13 reversals in the
staircase.’)

\subsection{slide-47}
\citet{webster:2012_color} reasoned like this:
\begin{enumerate}
\item Where the two diagonals are from within the same colour category,
the grouping midpoint colour should be the midpoint in a colour space that
represents retinal colour discrimination.
\item Where the two diagonals are from different colour categories (one is
green, the other is blue), and one of the diagonals is just over a boundary,
then the grouping midpoint colour should be closer to the colour of the diagonal
that is just over the boundary in a colour space that
represents retinal colour discrimination.
\end{enumerate}

This prediction is illustrated in the figure.
The dotted line shows the prediction if there is no effect of colour category
on perceptual grouping; the solid lines show different strengths of effect.

They found that almost no indicators of an
effect of colour category on perceptual grouping
(there are a variety of measures and one was signifiant:
p. 381: ‘the participants’ settings thus trended toward a (very weak) CP effect,
with an average bias of 0.10 (which was nevertheless significantly differ-
ent from zero; t (7) = 3.73, p < .01).’).
They also suggested that these effects may be due to the use of
CIELAB as a colour space
(p. 382: ‘ the small biases we found in the observer’s settings may in part
include an artifact of the stimulus space, weakening further the evidence for a
clear CP effect in the grouping task.’)

\subsection{slide-48}
‘In the hue scaling experiment observers are typically shown a set
of colors one at a time, and for each report the relative proportions of each
primary they perceive.’

\citet{webster:2012_color} did an analysis of other studies and found
an effect across the blue-green category boundary.
Roughly speaking, native English speakers’ tendancy to say how much
green or blue is in a colour cannot be predicted by its position in a
cone-opponent colour space (one based on retinal colour processing)
but also appears to be influenced by categories.
(p. 385: ‘The prediction for the average curve ... indicates a CP bias of
0.23 across the blue–green boundary. This ... underestimates the degree of
bias in individual observer’s settings, which spanned a very wide range
from 0.02 to 0.92 with a mean of 0.35 ... The distribution of categorical
biases in the hue settings is substantially different from 0 (t (58) =
15.79, p < 0.0001).’)

I don’t think this is very informative for two reasons.
First, it seems to me that hue scaling could be influenced by tendancies to
label a colour. The judgements aren’t obviously a reflection of how things
appear.

Second, colour discrimination may not be well controlled in this study.
(There are also issues about the blue-green boundary: the effects could in
principle be due to the fact that the boundary ‘falls close to the L pole
of the LvsM axis of cone-opponent space (Malkoc et al., 2005)’
\citep[p.~388]{webster:2012_color} .)

\subsection{slide-51}
The point of all of these experiments is to see whether differences with
respect to the properties denoted by colour terms such as ‘red’ affect visual
appearances.
Mostly the findings suggest that they do not.

\subsection{slide-52}
So do red things differ in visual appearance from non-red things?

\subsection{slide-53}
While doubts might be raised about the details of one or another method, the
overall pattern is clear: the most careful attempts to find differences in
appearance associated with properties denoted by colour terms like ‘red’ have
all failed.

To reject this conclusion, we would have to insist that differences in
appearance exist but influence neither judgements of perceptual similarity nor
perceptual grouping and so are too subtle to detect.
This is certainly possible, but it seems wrong simply to insist that it is correct.

\subsection{slide-54}
Whether colour terms like ‘red’ denote properties of objects that are
presented in visual experience is something best decided experimentally, not
introspectively; and, as far as we know, they do not.

\subsection{slide-52}


\section{Why Do Some Claim to Visually Experience Red?}

Suppose, as argued, it is untrue that humans visually experience red or any
other categorical colour properties.
Why have so many philosophers have assumed the opposite, and done so without argument?

\subsection{slide-55}
Suppose, as argued, it is untrue that humans visually experience red or any
other categorical colour properties.
Why have so many philosophers have assumed the opposite, and done so without argument?

\subsection{slide-56}
Some time  after (but not usually immediately after)  learning to use a colour term like
‘red’ somewhat accurately, humans become faster and more accurate at
distinguishing things which differ in whether they have the property denoted by
that colour term (faster: \citealp{Bornstein:1984cb}; more accurate:
\citealp[p.\ 22--7]{Roberson:1999rk}; not usually immediately:
\citealp{Franklin:2005hp}). In fact, methods highly similar to those which
indicate the absence of appearances do reveal that these properties affect
speed and accuracy of discrimination (\citealp{witzel:2014_categorical}). As
discrimination of these colour properties depends on pre-attentive processes
which are automatic in some of the senses that perceptual processes are
\citep{Daoutis:2006ij,clifford_color_2010}, the abilities to discriminate may
intuitively give rise to the impression that properties like \emph{red} affect
how things appear.

\subsection{slide-57}
How?
How could these pre-attentive, automatic abilities to discriminate
give rise to the impression that properties like \emph{red} affect
how things appear?

\subsection{slide-58}
Option 1: The processes of discrimination modulate the overall phenomenal
character of experience, and do so differently depending on which
categorical colour properties are discriminated.
On this option, we have something like a phenomenal signal of sameness and
difference.

\subsection{slide-59}
Option 2: Philosophers (and perhaps others) intuitively (and incorrectly) assume that
capacities to discriminate depend on how things visuall appear.
That is, the intuitive (and incorrect) assumption is that
I can visually distinguish categorical properties because things
visually appear to have those categorical colour properties.
(Compare  \citep[pp.~288--9]{Bornstein:1987vv}:
“Discriminable wavelengths seem to be categorized together because they appear perceptually similar”.)


%--- end paste
%---------------





\bibliography{$HOME/endnote/phd_biblio}



\end{document}
