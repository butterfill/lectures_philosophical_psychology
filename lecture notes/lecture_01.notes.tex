 %!TEX TS-program = xelatex
%!TEX encoding = UTF-8 Unicode

%\def \papersize {a5paper}
\def \papersize {a4paper}
%\def \papersize {letterpaper}

%\documentclass[14pt,\papersize]{extarticle}
\documentclass[12pt,\papersize]{extarticle}
% extarticle is like article but can handle 8pt, 9pt, 10pt, 11pt, 12pt, 14pt, 17pt, and 20pt text

\def \ititle {Origins of Mind: Lecture Notes}
\def \isubtitle {Lecture 01}
%comment some of the following out depending on whether anonymous
\def \iauthor {Stephen A.\ Butterfill}
\def \iemail{s.butterfill@warwick.ac.uk% \& corrado.sinigaglia@unimi.it
}
%\def \iauthor {}
%\def \iemail{}
%\date{}

%\input{$HOME/Documents/submissions/preamble_steve_paper4}
\input{$HOME/latex_imports/preamble_steve_lecture_notes}

%no indent, space between paragraphs
\usepackage{parskip}

%comment these out if not anonymous:
%\author{}
%\date{}

%for e reader version: small margins
% (remove all for paper!)
%\geometry{headsep=2em} %keep running header away from text
%\geometry{footskip=1.5cm} %keep page numbers away from text
%\geometry{top=1cm} %increase to 3.5 if use header
%\geometry{bottom=2cm} %increase to 3.5 if use header
%\geometry{left=1cm} %increase to 3.5 if use header
%\geometry{right=1cm} %increase to 3.5 if use header

% disables chapter, section and subsection numbering
\setcounter{secnumdepth}{-1}

%avoid overhang
\tolerance=5000

%\setromanfont[Mapping=tex-text]{Sabon LT Std}


%for putting citations into main text (for reading):
% use bibentry command
% nb this doesn’t work with mynewapa style; use apalike for \bibliographystyle
% nb2: use \nobibliography to introduce the readings
\usepackage{bibentry}

%screws up word count for some reason:
%\bibliographystyle{$HOME/Documents/submissions/mynewapa}
\bibliographystyle{apalike}


\begin{document}



\setlength\footnotesep{1em}






%---------------
%--- start paste
\title {Philosophical Psychology \\ Lecture 01: Intention and Motor Representation in Purposive Action}
 
 
 
\maketitle
 
\subsection{title-slide}
Any attempt to bring scientific discoveries about action into a philosophical
discussion quickly runs into a significant obstacle ...
[The obstacle will be apparently completely different way of thinking about action.]
 
\subsection{slide-3}
Which events are actions?
In philosophy, answering this question would typically answered by appeal to intention
or practical reasoning.
 
Such views tend to be neutral on how
the attitudes and processes ultimately connect to bodily movements;
that is considered to be merely an implementation detail ...
 
They are neutral in this sense: the views do not depend in any way on facts about that
distinguish one kind of body from another, or on facts about how the body’s movements
are ultimately controlled ...
 
\subsection{slide-4}
In cognitive science ... little to say about actions whose purposes
involve things the motor system doesn’t care about---your motor system
doesn’t care whether the plane you are stepping is headed for Milan or
for Rome, but this sort of difference can affect whether your actions succeed or fail.
 
--------
\subsection{slide-5}
You might just say that the two disciplines are talking past each other,
or you might say that they are offering two complementary but independent
models of action.
 
Call this the ‘Two Stories View’ (or divorced, but living together).
 
But we want to argue that these two views are components of a single, larger story about action.
Although intention and motor representation can usefully be studied in isolation
to some extent, a full understanding of action will require understanding
interfaces between the two …
 
But first let me fill in a little detail about each of the stories ...
 
\subsection{slide-6}
Intentions and motor representations belong to quite different enterprises, I think.
 
WHY MOTOR REPRESENTATION? ONE ADVANTAGE OF THEORISING ABOUT MOTOR REPRESENTATIONS IS
THAT YOU CAN GENERATE TESTABLE PREDICTIONS. INTENTIONS ARE FICTIONS. YOU CANNOT GENERATE
TESTABLE PREDICTIONS FROM A THEORY ABOUT INTENTION. SO IF YOUR QUESTION IS ABOUT WHAT
ENABLES COOPERATION, YOU WILL PROBABLY NEED A STORY ABOUT PSYCHOLOGICAL MECHANISMS.
 
ONTOLOGY : YOU WOULDN’T DO IT WITHOUT THINKING ABOUT PHYSICS, WOULD YOU?
SOCIAL ONTOLOGY NEEDS PSYCHOLOGY LIKE ONTOLOGY NEEDS PHYSICS.
 
\subsection{slide-9}
Gilbert is explicit about this.
What grounds her theorising>
 
This is fine. But it doesn't involve anything measurable or repeatable.
You will never be able to generate any testable predictions.
So you are quite unlikely to be on the path to discovering about how things work.
 
Which is fine if you’re Gilbert, of course, because she has no such aim.
But if you’re me, you want to know how things work.
 
\subsection{intro\_to\_intention}
 
 
\section{Intention}
 
What are intentions and what is their role in practical reasoning and action? Some basic
distinctions from the Philosophy of Action can give us a fix on the notion of intention as
many philosophers conceive of it.
 
\subsection{slide-11}
You tilt the bottle thereby pouring prosecco all over Zac's trousers.
You might say, the goal of my actions was not to soak Zac's trousers but to
fill his glass.
 
\subsection{slide-12}
As this illustrates,
some actions involving are purposive in the sense that
 
\subsection{slide-13}
among all their actual and possible consequences,
 
\subsection{slide-14}
there are outcomes to which they are directed
 
In such cases we can say that the actions are clearly purposive.
 
\subsection{slide-15}
Concerning any such actions, we can ask
What is the relation between a purposive action and the outcome or outcomes to which it is directed?
 
\subsection{slide-16}
The standard answer to this question involves intention.
 
\subsection{slide-17}
An intention (1) specifies an outcome,
 
\subsection{slide-18}
(2) coordinates the one or several activities which comprise the action;
 
and (3) coordinate these activities in a way that would normally facilitate the outcome’s occurrence.
 
What binds particular component actions together into larger purposive actions?
It is the fact that these actions are all parts of plans involving a single intention.
What singles out an actual or possible outcome as one to which the component
actions are collectively directed? It is the fact that this outcome is
represented by the intention.
 
So the intention is what binds component actions together into purposive actions and
links the action taken as a whole to the outcomes to which they are directed.
 
\subsection{slide-19}
But is intention the only thing that can link actions to outcomes?
I will suggest that motor representations can likewise perform this role.
 
\subsection{slide-20}
Any questions or remarks?
 
I wonder if anyone wants to say something about this story and whether it answers the question,
What is the relation between a purposive action and the outcome or outcomes to which it is directed?
(I think it does and no one has so far objected to it except on very broad theoretical grounds
involving Anscombe.)
 
Or maybe there’s a basic question about intention.
 
\subsection{intro\_to\_motor\_representation}
 
 
\section{Motor Representation}
 
Motor representations are involved in performing and preparing actions.
Not all representations represent patterns of joint displacements and bodily
configurations: some represent outcomes such as the grasping of an object, which
may be done in different ways in different contexts.
 
\subsection{slide-22}
Let me go back and start with some almost uncontroversial facts about motor representations and
their action-coordinating role.
 
Why postulate motor representations at all? [Dependence of present actions on future actions
is one reason for doing so.]
 
\subsection{slide-23}
Suppose you are a cook who needs to take an egg from its box, crack it and put it (except for the
shell) into a bowl ready for beating into a carbonara sauce.
Even for such mundane, routine actions, the constraints on adequate performance can vary
significantly depending on subtle variations in context. For example, the position of a hot pan
may require altering the trajectory along which the egg is transported, or time pressures may mean
that the action must be performed unusually swiftly on this occasion.
Further, many of the constraints on performance involve relations between actions occurring at
different times.
To illustrate, how tightly you need to grip the egg now depends, among other things, on the forces
to which you will subject the egg in lifting it later.
It turns out that people reliably grip objects such as eggs just tightly enough across a range of
conditions in which the optimal tightness of grip varies.
This indicates (along with much other evidence) that information about the cook’s anticipated
future hand and arm movements appropriately influences how tightly she initially grips the egg
(compare \citealp{kawato:1999_internal}).
This anticipatory control of grasp,
like several other features of action performance (\citealp[see][chapter 1]{rosenbaum:2010_human} for more examples),
is not plausibly a consequence of mindless physiology, nor of intention and practical reasoning.
This is one reason for postulating motor representations, which characteristically play a role in
coordinating sequences of very small scale actions such as grasping an egg in order to lift it.
 
[The scale of an actual action can be defined in terms of means-end relations.
Given two actions which are related as means to ends by the processes and representations
involved in their performance, the first is smaller in scale than the second just if the
first is a means to the second. Generalising, we associate the scale of an actual action
with the depth of the hierarchy of outcomes that are related to it by the transitive closure
of the means-ends relation. Then, generalising further, a repeatable action (something that
different agents might do independently on several occasions) is associated with a scale
characteristic of the things people do when they perform that action. Given that actions
such as cooking a meal or painting a house count as small-scale actions, actions such as
grasping an egg or dipping a brush into a can of paint are very-small scale. Note that we
do not stipulate a tight link between the very small scale and the motoric. In some cases
intentions may play a role in coordinating sequences of very small scale purposive actions,
and in some cases motor representations may concern actions which are not very small scale.
The claim we wish to consider is only that, often enough, explaining the coordination of
sequences of very small scale actions appears to involve representations but not, or not
only, intentions. To a first approximation, \emph{motor representation} is a label for
such representations.%
\footnote{%
Much more to be said about what motor representations are; for instance, see \citet{butterfill:2012_intention} for the view that motor representations can be distinguished by representational format.
}]
 
\subsection{slide-24}
‘a given motor act may change both as a function of what motor act will follow it—a sign of
planning—and as a function of what motor act preceded it—a sign
of memory’ \citep[p.~294]{cohen:2004_wherea}.
 
\subsection{slide-25}
What do motor representations represent? An initially attractive, conservative
view would be that they represent bodily configurations and joint displacements,
or perhaps sequences of these, only.
 
\subsection{slide-26}
However there is now a significant body of evidence that some motor representations
do not specify particular sequences of bodily configurations and joint displacements,
but rather represent outcomes such as the grasping of an egg or the pressing of a switch.
These are outcomes which might, on different occasions, involve very different bodily
configurations and joint displacements
(see \citealp{rizzolatti_functional_2010} for a selective review).
 
Such outcomes are abstract relative to bodily configurations and joint displacements
in that there are many different ways of achieving them.
 
\subsection{slide-28}
Since this is important, let me pause for a moment.
To say that something is an outcome is not to say very much: merely that it is a
possible or actual state of affairs which is brought about in some way by an action.
 
What I mean is that motor representations represent the sort of outcomes that are plausibly
thought of as goals of actions, things such as transporting and breaking an egg,
or flipping a switch.
 
This kind of outcomes are quite different from both patterns of joint displacement and
bodily movements and also from end states.
 
\subsection{slide-29}
In my terms, a goal is merely outcome.
Contrast goals with goal-states, which are mental or functional states which specify goals.
A motor representation is a goal-state; the goal is the thing it represents.
 
\subsection{slide-30}
Motor schema are just motor representations which specify incomplete outcomes.
[Of course schema are incomplete, so do not specify outcomes but outcome types.]
 
Motor schema are ‘internal models or stored representations that represent generic knowledge about
a certain pattern of action and are implicated in the production and control of action’
\citep{mylopoulos:2016_intentions}.
 
‘Motor schemas are more abstract and stable representations of actions than motor
representations.’
 
‘They are internal models or stored representations that represent generic knowledge
about a certain pattern of action and are implicated in the production and control of
action. For instance, in the influential Motor Schema Theory proposed by Richard Schmidt
(1975, 2003), a motor schema involves a generalized motor program, together with corre-
sponding ‘recall’ and ‘recognition’ schemas. The generalized motor program is thought to
contain an abstract representation defining the general form or pattern of an action,
that is the organization and structure common to a set of motor acts (e.g., invariant
features pertaining to the order of events, their spatial configuration, their relative
timing and the relative force with which they are produced). This generalized motor
program has parameters that control it. In order to determine how an action should be
performed on a given occasion, parameter values adapted to the situation must be
specified. Thus, a motor schema also includes a rule or system of rules describing the
relationships between initial conditions, parameter values and outcomes and allowing us
to perform the action over a large range of conditions (the ‘recall schema’ in Schmidt’s
terminology). Finally, the motor schema also includes a rule or system of rules
describing the relationships between initial conditions, exteroceptive and propri-
oceptive sensory feedback during an action, and action outcome (a recognition sche- ma),
allowing agents to know when they have made an error – i.e., the action does not have
the sensory consequences it is expected to have – and to correct for it.’
 
Are motor schemas an alterantive to motor representations? Tempting to think that where
they talk about motor schema, we think of motor representations that represent outcomes
that are relatively distal from action (e.g. are affector-neutral).
 
This would explain why they contrast motor schema with motor programmes. We use ‘motor
representation’ to include both.
 
\subsection{slide-31}
A goal represented motorically triggers a process which, via computations of things
like end states, starting states and smoothness, eventually results in joint
displacements; and when things go well, these joint displacements together with the
resulting bodily configurations bring about, or constitute, the occurrence of the goal.
 
\subsection{slide-32}
But of course this is a simplification. Motor representations can trigger processes
which result in further goals being represented, as for example when a motor
representation of the transporting of an object triggers representations of reaching,
grasping, placing and releasing.
 
\subsection{slide-33}
The processes linking motor representations are planning like in two respects:
(i) means-end ...
 
\subsection{slide-34}
... and (ii) relational constraints
 
\subsection{slide-35}
But how do we know that motor representations carry information about such outcomes?
I’m glad you asked, let me explain ...
 
\subsection{slide-36}
If you were to observe someone phi-ing,
then motor representations would occur in you
much like those that would occur in you if it were you, not her, who was phi-ing.
 
This will be a focus of interest in a later session.
For now it’s just a handy fact that simplifies testing.
 
\subsection{slide-38}
For a quick illustration of how we know about the double life of motor representation, consider
this experiment ....
 
‘Double TMS pulses were applied just prior to stimuli presentation to selectively prime the cortical
activity specifically in the lip (LipM1) or tongue (TongueM1) area’
\citep[p.~381]{dausilio:2009_motor}
 
‘We hypothesized that focal stimulation would facilitate the perception of
the concordant phonemes ([d] and [t] with TMS to TongueM1), but that
there would be inhibition of perception of the discordant items
([b] and [p] in this case). Behavioral effects were measured via reaction
times (RTs) and error rates.’ \citep[p.~382]{dausilio:2009_motor}
 
\subsection{slide-40}
‘The posterior section of the frontal lobe contains the motor areas,’ (p.~4)
Now you know as much about the brain as I do.
 
Mention primary and supplementary motor areas : we use the term ‘motor’ loosely
(compare ‘visual’, which also has narrower and broader uses in
neuroscience).
 
\subsection{slide-41}
[end of aside on the double life of motor representation]
 
 
\subsection{slide-44}
TMS to measure MEP
 
\subsection{slide-45}
They also had an occluded end version ...
 
\subsection{slide-46}
Incidentally, ‘the observed direction of the modulation was not consistent with previous TMS
literature. Specifically, MEP amplitudes were significantly lower in the Object-Present than in the
Object-Absent conditions (Fig. 2), suggesting that there was an inhibitory effect of object
manipulation on the activity of M1 during action observation.’
 
\subsection{slide-50}
Umiltà et al, 2008 : single cell recordings in monkeys
 
MEPs (TMS amplified) in humans
 
\subsection{slide-51}
TMS MEP, humans.
 
Shown video, then a static picture.
Is this the same goal as you saw in the video?
Press one of two keys.
‘They were explicitly told to ignore the effector and make a judgment on the type of act only.’
 
\subsection{slide-52}
Key finding: TMS to both ventral premotor cortex (PMv) and left supramarginal gyrus (SMG)
increases RTs regardless of whether it’s the same effector or a different effector.
(You can’t see same/different effector in this figure.)
 
KEY: superior temporal sulcus (STS), and a parietofrontal system consisting of the intraparietal
sulcus (IPS) and inferior parietal lobule (IPL) plus the ventral premotor cortex (PMv) and caudal
part of inferior frontal gyrus (IFG). In some instances also, the superior parietal lobule (SPL)
 
\subsection{slide-53}
By contrast, TMS to superior temporal sulcus (STS) increased RT only for judgements
where the video effector was the same as the photo effector.
 
\subsection{slide-54}
The experiments providing such evidence typically involve a marker of motor representation,
such as a pattern of neuronal firings, a motor evoked potential or a behavioural performance
profile, which, in controlled settings, allows sameness or difference of motor representation
to be distinguished. Such markers can be exploited to show that the sameness and difference
of motor representation is linked to the sameness and difference of an outcome such as the
grasping of a particular object.
(Pioneering uses of this method include \citealp{rizzolatti:1988_functional,Rizzolatti:2001ug};
it has since been developed in many ways: see, for example,
\citet{hamilton:2008_action, cattaneo:2009_representation, cattaneo:2010_state-dependent,
rochat:2010_responses, bonini:2010_ventral, koch:2010_resonance}.)
 
\subsection{slide-60}
To illustrate, consider a sequence of actions which might be involved in shoplifting an apple: you have to secure the apple, transport it, and position it in your pocket.
Each of these outcomes can be represented motorically.
 
\subsection{slide-61}
Motor processes are planning-like in that they involve computing means from ends.
Thus a representation of an end like securing it [the apple] can trigger a process
that results in the representation of outcomes that are means to this end.
 
\subsection{slide-62}
Motor processes are also planning-like in that which means are selected in preparing an
action that will occur early in the sequence may affect needs that will arise only later
in a later part of the actions.
For instance, how the apple is grasped at an early point in the sequence may be determined
in part by what would be a more comfortable way for the other hand to grasp it later.
 
\subsection{slide-63}
So motor representations of outcomes guide planning-like processes.
This is why I think it’s not just that they carry information about outcomes
like grasping an apple, but that they also represent such outcomes.
 
\subsection{motor\_representations\_ground\_goals}
 
 
\section{Motor Representations Ground the Directedness of Actions to Goals}
 
How do intentions ground the purposiveness of actions? On any standard view, an intention
represents an outcome, causes an action, and does so in a way that would normally facilitate
the outcome’s occurrence. Similarly, some motor representations represent action
outcomes, play a role in generating actions, and do this in a way that normally facilitates
the occurrence of the outcomes represented. Like intentions, motor representations
ground the directedness of actions to outcomes which are thereby goals of the actions.
 
\subsection{slide-66}
 
\subsection{slide-67}
Now as Elisabeth Pacherie has argued (and I’ve had a go at arguing this in joint work with Corrado Sinigaglia recently too),
motor representations are relevantly similar to intentions.
Of course motor representations differ from intentions in some important ways (as Pacherie also notes).
But they are similar in the respects that matter for explaining the purposiveness of action.
(1) Like intentions, some motor representations represent outcomes (and not merely patters of joint displacement, say).
(2) Like intentions, some motor representations play a role in coordinating multiple more component activities by virtue of their role as elements in hierarchically structured plans.
(3) And, like intentions, some motor representations coordinate these activities in a way that would normally facilitate the outcome’s occurrence.
The claim is not that \emph{all} purposive actions are linked to outcomes by motor representations, just that some are.
In some cases, the purposiveness of an action is grounded in a motor representation of an outcome; in other cases it is grounded in an intention.
And of course in many cases it may be that both intention and motor representation are involved.
 
\subsection{slide-68}
I started by observing that there are two quite different approaches to
answering the question, Which events are actions?
What does the ground I’ve so far tell us about this?
 
I don’t think it tells us much yet (there’s much more to come).
 
First, both stories are stories about purposive actions.
So they don’t have clearly demarcated domains.
 
Second, it may make it tempting to think that motor representations are just a kind of intention,
and so to take the radical view that there are not really two distinct stories here at all.
 
\subsection{motor\_representations\_arent\_intentions}
 
 
\section{Motor Representations Aren’t Intentions}
 
Explains why motor representations aren’t intentions.
 
\subsection{slide-72}
As background we first need a generic distinction between content and format.
Imagine you are in an unfamiliar city and are trying to get to the central station.
A stranger offers you two routes. Each route could be represented by a distinct line
on a paper map. The difference between the two lines is a difference in content.
 
\subsection{slide-73}
Each of the routes could alternatively have been represented by a distinct series
of instructions written on the same piece of paper; these cartographic and
propositional representations differ in format. The format of a representation
constrains its possible contents. For example, a representation with a cartographic
format cannot represent what is represented by sentences such as `There could not be a
mountain whose summit is inaccessible.'\footnote{ Note that the distinction between
content and format is orthogonal to issues about representational medium. The maps in
our illustration may be paper map or electronic maps, and the instructions may be spoken,
signed or written. This difference is one of medium.} The distinction between content and
format is necessary because, as our illustration shows, each can be varied independently
of the other.
 
\subsection{slide-74}
Format matters because only where two representations have the same format can they be straightforwardly inferentially integrated.
 
To illustrate, let’s stay with representations of routes.
Suppose you are given some verbal instructions describing a route. You are then shown a representation of a route on a map and asked whether this is the same route that was verbally described. You are not allowed to find out by following the routes or by imagining following them.
Special cases aside, answering the question will involve a process of translation because two distinct representational formats are involved, propositional and cartographic. It is not be enough that you could follow either representation of the route. You will also need to be able to translate from at least one representational format into at least one other format.
 
\subsection{slide-75}
How in general can we identify or distinguish representational formats? Because representational formats are typically associated with characteristic performance profiles, it is sometimes possible to infer similarities and differences in representational format from similarities and differences in the processes in which representations feature.
 
To illustrate, suppose that you have a route representation and I want to work out whether it this representation has a cartographic or propositional format. One way to do this might be to test your performance on different tasks. If the representation is propositional you are likely to be relatively fast at identifying key landmarks but relatively slow at translating the route into a sequence of compass directions; but the converse will be true if your representation is cartographic.
 
\subsection{slide-76}
The same principle---distinguishing and identifying formats by measuring characteristic processing
profile---works for mental representations too.
 
To illustrate, compare imagining seeing an object moving with actually seeing it move.
For this comparison we need to distinguish two ways of imagining seeing. There is a way of
imagining seeing which phenomenologically is something like seeing except that it does not
necessarily involve being receptive to stimuli. This way of imagining seeing, sometimes
called `sensory imagining', is commonly distinguished from cognitive ways of imagining
seeing which might for example involve thinking about seeing.
It is this way of imagining seeing an object move that we wish to compare with actually
seeing an object move.
 
\subsection{slide-77}
Imagining seeing an object move and actually seeing an object move have similarities in
characteristic performance profile. For instance, whether an object can be seen all at once depends
on its size and distance from the perceiver; strikingly, when subjects imagine seeing an object,
whether they can imagine seeing it all at once depends in the same way on size and distance
(\citealp{kosslyn:1978_measuring}; \citealp[p.\ 99ff]{kosslyn:1994_image}).
 
Also, how long it takes to imagine looking over an object depends on the object's subjective size in
the same way that how long it would take to actually look over that object would depend on its
subjective size \citep{kosslyn:1978_visual}.
 
The similarities in characteristic performance profile and the particular patterns of interference
are good (if non-decisive) reasons to conjecture that imagining seeing and actually seeing involve
representations with a common format.
 
\subsection{slide-78}
One way of imagining action is phenomenologically something like acting except
that such imaginings are not necessarily responsive to the features of actual
objects and do not necessarily result in bodily movements.
 
There is evidence that the way imagining performing an action unfolds in time is
similar in some respects to the way actually performing an action of the same type would unfold.
 
For instance, how long it takes to imagine moving an object is closely related to
how long it would take to actually move that object \citep{decety:1989_timing,
decety:1996_imagined, Jeannerod:1994oz}.
 
In addition, for actions such as grasping the handle of a cup, manipulating the
target object in ways that would make the action harder (such as orienting the cup's
handle to make it less convenient for you to grasp) make a corresponding difference to
the effort involved in imagining performing the action \citep{parsons:1994_temporal,
frak:2001_orientation}.
 
\subsection{slide-79}
Contrast imagining rotating a ball with imagining seeing a ball rotate.
 
As is implied by what we’ve already said, these have quite different characteristic performance
profiles.
 
How quickly the former can be done is a function of how long it would take the agent to rotate the
ball, whereas how quickly the latter can be done depends on how rapidly the ball can rotate and
still be perceived as rotating.
 
Further, in some cases rotating a ball clockwise is easier than rotating it anti-clockwise, and so
is imagining a ball rotate. By contrast, the effort involved in actually seeing or imagining seeing
a ball rotate does not similarly differ depending on direction.
 
\subsection{slide-80}
It may be objected that performance differences such as these can be explained without appealing to
a difference in format. After all, rotating a ball involves an action whereas a ball rotating does
not; consequently, imagining the former may be thought to differ from imagining the latter with
respect to the contents of the representations involved. Supposing that there are differences in
content here and in other cases, could these fully explain differences in performance profile? To
see why not, consider two tasks involving mental rotation. Judging the laterality of a rotated
letter is thought to involve phenomenologically vision-like imagination
\citep{jordan:2001_cortical}, whereas judging the laterality of a rotated hand is thought to involve
phenomenologically action-like imagination \citep{parsons:1987_imagined, gentilucci:1998_right}.
Ordinary subjects who are asked to judge the laterality of a hand rotated to various degrees are
less accurate when the hand's position is biomechanically awkward. By contrast, no such effect
occurs for comparable tasks involving letters rather than hands. How could this difference in
performance in imagining hands and letters be explained? Consider the claim that the difference in
performance can be fully explained by a difference in the content of the representations involved.
Initially this might seem plausible because one task involves hands whereas the other involves
letters. However, there are subjects who can perform both tasks but whose performance is not
different for hands and letters \citep{Fiori:2012fk}. These are subjects suffering Amyotrophic
Lateral Sclerosis (ALS), which impairs motor representation \citep{parsons:1998_cerebrally}. Since
ALS and ordinary subjects encounter the same stimuli and perform the same tasks, there seems to be
no reason (other than our hypothesis about a difference in format) to suppose that the two groups'
performance involves representations with different contents. So if the hand-letter difference in
performance were entirely explained by a difference in content, we would expect ALS and ordinary
subjects to exhibit the same difference in performance. But this is not the case. This is an
obstacle to supposing that the hand-letter difference in performance in ordinary subjects could be
explained by appeal to content.
 
\subsection{slide-81}
So far we have been arguing that motor and visual representations differ in format. Why suppose that
motor representations also differ in format from intentions? Contrast two ways of imagining taking a
shot in basketball, one involving the phenomenologically action-like kind of imagination and the
other involving a cognitive kind of imagination. The contrast we require is roughly between the way
a former player might imagine this and the way that someone who has only ever read about basketball
might imagine it. As we have seen, the way phenomenologically action-like imagination unfolds in
time and the amount of effort it involves will depend on bio-mechanical, dynamical and postural
constraints, among others. These constraints are closely related to those which govern actually
performing such actions \citep{Jeannerod:2001yb}, and some can be altered by acquiring or losing
motor expertise. By contrast no such constraints would be expected always to apply where a cognitive
kind of imagination is involved. In line with the general strategy of inferring differences in
format from differences in characteristic performance profile, we conclude that motor
representations differ in format from those involved in cognitive kinds of imagination, which are
plausibly propositional.
 
\subsection{slide-82}
\begin{enumerate}
\item Only representations with a common format can be inferentially integrated.
\item Any two intentions can be inferentially integrated in practical reasoning.
\item My intention that I visit the ZiF is a propositional attitude.
\end{enumerate}
Therefore:
\begin{enumerate}[resume]
\item All intentions are propositional attitudes
\end{enumerate}
But:
\begin{enumerate}[resume]
\item No motor representations are propositional attitudes.
\end{enumerate}
So:
\begin{enumerate}[resume]
\item No motor representations are intentions.
\end{enumerate}
 
\subsection{slide-86}
So my concern was to argue that motor representations aren’t intentions.
 
\subsection{slide-87}
So where does this leave us with respect to our starting point,
the ‘Two Stories’ view?
 
On the face of it, everything I’ve said so far is compatible with that View
and might even be taken to support it.
 
But there is a problem ...
 
\subsection{slide-88}
Recall stealing an apple ...
 
\subsection{slide-89}
Imagine you are a hungry shoplifter with a powerful desire for an apple. After carefully
considering the pile of apples on the stall, you identify one that can be discretely
snatched and form an intention to steal that apple as you casually saunter past the stall.
In grasping, transporting and pocketing the apple your movements are controlled by motor
representations of these outcomes. So your theft depends on an intention, on some motor
representations and on there being a match between the outcome specified by the intention
and the outcomes specified by the motor representations.
 
In short: practical reasoning and motor processes are part of a single story about the
performance of action; there must be non-accidental matches between the contents of
intentions and motor representations.
 
\textbf{So we can’t accept the Two Stories View because there must sometimes be content-respecting
causal interactions involving intentions and motor representations.
Without these, our intentions and motor representations would never non-accidentally
have synergistic effects on our actions.}
 
\subsection{the\_interface\_problem}
 
 
\section{The Interface Problem}
 
For a single action, which outcomes it is directed to may be multiply
determined by an intention and, seemingly independently, by a motor
representation. Unless such intentions and motor representations are to pull
an agent in incompatible directions, which would typically impair action
execution, there are requirements concerning how the outcomes they represent
must be related to each other. This is the interface problem: explain how any
such requirements could be non-accidentally met.
 
\subsection{slide-91}
Imagine that you are strapped to a spinning wheel facing near certain death as it plunges you into
freezing water. To your right you can see a lever and to your left there is a button. In deciding
that pulling the lever offers you a better chance of survival than pushing the button, you form an
intention to pull the lever, hoping that this will stop the wheel. If things go well, and if
intentions are not mere epiphenomena, this intention will result in your reaching for, grasping and
pulling the lever. These actions---reaching, grasping and pulling---may be directed to specific
outcomes in virtue of motor representations which guide their execution. It shouldn't be an accident
that, in your situation, you both intend to pull a lever and you end up with motor representations
of reaching for, grasping and pulling that very lever, so that the outcomes specified by your
intention match those specified by motor representations. If this match between outcomes variously
specified by intentions and by motor representations is not to be accidental, what could explain it?
 
As we have just seen, motor representations specify goals.
 
\subsection{slide-92}
And of course, so do intentions.
 
\subsection{slide-93}
Further, many actions involve both intention and motor representation.
When, for example, you form an intention to turn the lights out, the goal
of flipping the light switch may be represented motorically in you.
 
The nonaccidental success of our actions therefore depends on the outcomes
specified by our intentions and motor representations matching.
 
\subsection{slide-94}
But how should they match?
I think they should match in this sense:
the occurrence of the outcome specified by the motor representation would
would normally constitute or cause, at least
partially, the occurrence of the outcome specified by the intention.
 
\subsection{slide-95}
Now we have to ask, How are nonaccidental matches possible?
If you asked a similar question about desire and intention, the answer would be
straightforward: desire and intention are integrated in practical reasoning, so it
is no surprise that what you intend sometimes nonaccidentally conforms to what you intend.
But we cannot give the same sort of answer in the case of motor representations and
intentions because ...
 
\subsection{slide-96}
Intention and motor representation are not inferentially integrated.
 
Beliefs, desires and intentions are related to the premises and conclusions
in practical reasoning. Motor representations are not.
Similarly, intentions do not feature in motor processes.
 
\subsection{slide-97}
Failure of inferential integration follows from the claim that they differ in
format and are not translated. But I suspect that more people will agree that there is a
lack of inferential integration than that they differ in representational format.
(must illustrate format with maps).
 
\subsection{slide-98}
So this is the Interface Problem: how do the outcomes specified by intentions and
motor representations ever nonaccidentally match?
 
\subsection{slide-100}
Having an intention is neither necessary nor sufficient.
 
\subsection{slide-101}
Jeannerod 2006, p. 12:
‘the term apraxia was coined by Liepmann to account for higher order motor
disorders observed in patients who, in spite of having no problem in executing
simple actions (e.g. grasping an object), fail in actions involving more
complex, and perhaps more conceptual, representations.’
 
Can people with ideomotor apraxia form intentions?
 
\subsection{slide-102}
\citep[p.~7]{mylopoulos:2016_intentions}: ‘Typically the result of lesion
to SMA or anterior corpus callosum, AHS is a condition in which patients
perform complex, goal-oriented movements with their cross-lesional limb
that they feel unable to directly inhibit or control. The limb is often
disproportionately reactive to environmental stimuli, carrying out
habitual behaviors that are inappropriate to the context, e.g., grabbing
food from a dinner companion’s plate (Della Sala 2005, 606). It is clear
from many of the behaviors observed in these cases that the anarchic limb
fails to hook up with the agent’s intentions.’
 
In the rest of this talk I’m not going to suggest a solution to the interface problem.
Instead, I want to mention some considerations
which may complicate attempts to solve it.
 
\subsection{slide-103}
There isn’t a big idea or single theme running through it.
I’m just going to present research that I’ve done in different areas---most or all
on action, on joint action and on mind reading.
 
But the Interface Problem stands exemplifies the sort of thing the course is about.
It’s how I think of philosophical psychology.
Reflection on psychological (and neuroscientific) discoveries raises a question
which is philosophically challenging in some way.
 
So the idea is to start with the discoveries and find broadly philosophical
questions arising from what has been discovered so far.
 
 
 
\section{Five Complications}
 
Any attempt to solve the interface problem must surmount at least five complications.
 
\subsection{slide-104}
This isn’t a preliminary to my talk; although I will say something about how to solve
the interface problem right at the end, I’m mainly concerned to persuade you that
the interface problem is tricky to solve.
 
\subsection{slide-105}
First consideration which complicates the interface problem: outcomes have a complex anatomy
comprising manipulation, target, form and more.
 
There is evidence that each of these can be represented motorically; and of course these
can all be specified by intentions too.
 
On the targets of actions, as Elisabeth has stressed,
motor representations represent not only ways of acting but also targets on which actions
might be performed and some of their features related to possible action outcomes involving
them (for a review see Gallese \& Sinigaglia 2011; for discussion see Pacherie 2000, pp.
410-3). For example visually encountering a mug sometimes involves representing features
such as the orientation and shape of its handle in motor terms (Buccino et al. 2009;
Costantini et al. 2010; Cardellicchio et al. 2011; Tucker \& Ellis 1998, 2001).
 
One possibility is that the Interface Problem breaks down into questions corresponding
to these three different components of outcomes. That is, an account of how the manipuations
specified by intentions and motor representations nonaccidentally match might end up being
quite different from an account of how the targets or forms match.
(I’m not saying this is right, just considering the possibility.)
 
\subsection{slide-106}
Second consideration which complicates the interface problem: scale.
This shows that we can’t think of the interface problem merely as a way of intentions
setting problems to be solved by motor representations: there may be multiple intentions
at different scales, and in some cases an intention may operate at a smaller scale than
a motor representation.
 
Suppose you have an intention to tap in time with a metronome.
Maintaining synchrony will involve two kinds of correction: phase and period shifts.
These appear appear to be made by mechanisms acting independently, so that correcting errors involves a distinctive pattern of overadjustment.
Adjustments involving phase shifts are largely automatic, adjustments involving changes in period are to some extent controlled.
 
How are period shifts controlled?
Importantly this is not currently known.
One possibility is that period adjustments can be made intentionally \citep[as][p.~2599 hint]{fairhurst:2013_being};
another is that there are a small number of ‘coordinative strategies’ \citep{repp:2008_sensorimotor} between which agents with sufficient skill can intentionally switch in something like the way in which they can intentionally switch from walking to running.
But either way, there can be two intentions: a larger-scale one to tap in time with a metronome
and a smaller-scale one to adjust the tapping which results in a period shift.
 
\subsection{slide-107}
EP: Skilled piano playing means being able to have intentions with respect to larger units than a
novice could manage. But in playing a 3-voice fugue you may need to pay attention to a particular
nger in order to keep the voices separate. So you need to be able to attend to both ‘large chunks’
(e.g. chords) of action and ‘small chunks’ (e.g. keypresses) simultaneously.
 
BACKGROUND:
Because no one can perform two actions without introducing some tiny variation between them, entrainment of any kind depends on continuous monitoring and ongoing adjustments \citep[p.~976]{repp:2005_sensorimotor}.
% \textcite[p.~976]{repp:2005_sensorimotor}: ‘A fundamental point about SMS is that it cannot be sustained without error correction, even if tapping starts without any asynchrony and continues at exactly the right mean tempo. Without error correction, the variability inherent in any periodic motor activity would accumulate from tap to tap, and the probability of large asynchronies would increase steadily (Hary \& Moore, 1987a; Voillaume, 1971; Vorberg \& Wing, 1996). The inability of even musically trained participants to stay in phase with a virtual metronome (i.e., with silent beats extrapolated from a metronome) can be demonstrated easily in the synchronization–continuation paradigm by computing virtual asynchronies for the continuation taps. These asynchronies usually get quite large within a few taps, although occasionally, virtual synchrony may be maintained for a while by chance.’
% \citet[p.~407]{repp:2013_sensorimotor}: ‘Error correction is essential to SMS, even in tapping with an isochronous, unperturbed metronome.’
One kind of adjustment is a phase shift, which occurs when one action in a sequence is delayed or brought forwards in time.
Another kind of adjustment is a period shift; that is, an increase or reduction in the speed with which all future actions are performed, or in the delay between all future adjacent pairs of actions.
These two kinds of adjustment,
phase shifts and period shifts,
appear to be made by mechanisms acting independently, so that correcting errors involves a distinctive pattern of overadjustment.%
\footnote{%
See \citet[pp.~474–6]{schulze:2005_keeping}. \citet{keller:2014_rhythm} suggest, further, that the two kinds of adjustment involve different brain networks.
Note that this view is currently controversial: \citet{loehr:2011_temporal} could be interpreted as providing evidence for a different account of how entrainment is maintained.
}
\citet[p.~987]{repp:2005_sensorimotor} argues, further, that while adjustments involving phase shifts are largely automatic, adjustments involving changes in period are to some extent controlled.
% (‘two error correction processes, one being largely automatic and operating via phase resetting, and the other being mostly under cognitive control and, presumably, operating via a modulation of the period of an internal timekeeper’ \citep[p.~987{repp:2005_sensorimotor})
One possibility is that period adjustments can be made intentionally \citep[as][p.~2599 hint]{fairhurst:2013_being};
another is that there are a small number of ‘coordinative strategies’ \citep{repp:2008_sensorimotor} between which agents with sufficient skill can intentionally switch in something like the way in which they can intentionally switch from walking to running.
 
\subsection{slide-108}
So it is not that intentions are restricted to specifying outcomes which form the head of the means-end
hierarchy of outcomes represented motorically.
They can also influence aspects of outcomes at smaller scales.
 
\subsection{slide-109}
Third consideration which complicates the interface problem: dynamics.
 
It’s ‘not just how motor representations are triggered by intentions, but how motor
representations’ sometimes nonaccidentally continue to match intentions as circumstances change in unforeseen ways ‘throughout
skill execution’
\citep[p.~19]{fridland:2016_skill}.
 
Here we need to distinguish different kinds of change.
Some changes can be flexibly accommodated motorically without any need for intention
to be involved, or even for the agents to be aware of the change. This includes peturbations
in the apparent direction of motion while drawing \citep{fourneret:1998_limited}.
But other changes may require a change in intention: circumstances may change in such a way
that you wish either to abandon the action altogether, or else switch target midway through.
 
\textbf{KEY}: in executing an intention you may learn something which causes you to change
your intention; for example, you may learn that the action is just too awkward, or that the
ball is out of reach. So motor processes can result in discoveries that nonaccidentally cause
changes in intention.
 
This also shows that we can’t think of the interface problem merely as a way of intentions
‘handing off’ to motor representations: in some cases, the matching of motor representations
and intentions will nonaccidentally persist.
 
\subsection{slide-110}
These reflections on dynamics (and on scale too[?]) suggest that the interface problem
is not a unidirectional but a bidirectional one. The agent who intends probably cannot be blind to
the ways in which motor representations structure her actions since the structure is both provides
and limits opportunities for interventions.
 
\subsection{slide-111}
The interface problem is the problem of explaining how there could be nonaccidental matches.
But there is a related developmental problem: What is the process by which humans acquire
abilities to ensure that their intentions and motor representations sometimes nonaccidentally
match?
 
A solution to the interface problem must provide a framework for answering the corresponding
question about development.
 
\subsection{slide-112}
Imagination: intentions and motor representations can nonaccidentally match not only
when we are acting but also when we are merely imagining acting.
 
\subsection{slide-114}
I started by observing that there are two quite different approaches to
answering the question, Which events are actions?
Philosophers mostly invoke intention, whereas neuroscientists and cognitive psychologists
focus on motor representation.
 
What are we to make of the disparity? Are the two simply talking about different phenomena?
I’ve offered two considerations which, taken together, suggest that they are not.
 
\subsection{slide-115}
The first consideration is that
 
Motor representations ground the directedness of actions to goals.
 
\subsection{slide-116}
The second consideration is that
 
Some actions involve both intention and motor representation.
 
\subsection{slide-117}
Given these two considerations, it seems to me that, in some cases, nonaccidentally successful
action requires that motor representations and intentions play a harmonious role in
guiding actions.
In particular, their contents must match.
 
But this requirement leads us to a question, which I’ve called ‘The Interface Problem’
 
How are non-accidental matches possible?
 
The interface problem is the problem of explaining how there could be nonaccidental matches
between motor representations and intentions.
 
In this talk I haven’t tried to solve the problem but merely mentioned five complications
which anyone who tries to solve it faces.
 
The interface problem is complicated by the \textbf{anatomy} of outcomes, the varying \textbf{scales}
at which intentions can specify actions, and by the possibility of nonaccidental \textbf{dynamical}
matching.
 
So while various researchers have proposed solutions to the Interface Problem -- Pacherie, Shepherd and
Butterfill and Sinigaglia, I’m doubtful that any are sufficient.
 
There’s also an obstacle I haven’t properly considered here yet.
As I said at the start, whereas motor representations are postulates of
scientific theories, intentions are, well, of less certain origin.
We know what motor representations are because they creatures of our own theories.
But talk about intention seems to be rooted in what philosophers think about
what Ben thinks about what is going on with Ayesha.
Maybe part of what makes the interface problem challenging is that we’re taking
intentions too seriously.
 
\subsection{action\_experience}
 
 
\section{Action Experience}
 
What do you experience when someone acts?
According to the Action Index Conjecture,
motor representations of outcomes structure
experiences, imaginings and (prospective) memories
in ways which provide opportunities for attention to actions directed to those outcomes.
Forming intentions concerning an outcome can influence attention to the action,
which can influence the persistence of a motor representation of the outcome.
 
Step back and consider the problem more generally. The mind is made up of lots of different, loosely
connected systems that work largely independently of each other. To a certain extent it’s fine for
them to go their own way; and of course since they all get the same inputs (what with being parts of
a single subject), there are limits on how separate the ways the go can be. Still, it’s often good
for them to be aligned, at least eventually.
 
Experience is what enables there to be nonaccidental eventual alignment of largely independent
cognitive systems. This is what experience is for.
 
Can we think along these lines in the case of action?
 
\textbf{What do we experience when someone acts?}
One possibility is that we experience only bodily configurations, joint displacements and
effects characteristic of particular actions.
For the purposes of this talk, I will assume that this is wrong.
Instead I will assume that we experience not only bodily configurations, joint displacements,
sounds and the rest but also goal-directed actions.
 
What might it mean to experience action?
 
\subsection{slide-119}
Could action experiences be like visual experiences?
This wound imply that there are expeirences of action which stand to motor representations
in something like the way that visual experiences stand
to visual representations.
 
\subsection{slide-120}
If this were right, how would it help with the interface problem?
 
\subsection{slide-121}
Visual representations cause visual experiences, which provide reasons for beliefs.
 
Beliefs (and desires) influence orientation and attention, and thereby visual representations.
 
This seems unlikely for several reasons.
First, vision involves a particular sensory modality. It would be quite radical to
postulate a motor modality.
 
Second, the interface problem in the case of vision is, I suppose, primarily about how
visual representations influence beliefs;
you don’t want influence in the other direction, or at least not too much.
By contrast, influence from intention to motor representation is essential;
so vision seems likely to provide a poor
for the case of action.
 
\subsection{slide-123}
what is an object index? Formally, an object index is ‘a mental token that functions as a pointer to
an object’ \citep[p.\ 11]{Leslie:1998zk}. If you imagine using your fingers to track moving objects,
an object index is the mental counterpart of a finger \citep[p.~68]{pylyshyn:1989_role}.
 
The interesting thing about object indexes is that a system of object
indexes (at least one, maybe more)
appears to underpin cognitive processes which are not
strictly perceptual but also do not involve beliefs or knowledge states.
While I can’t fully explain the evidence for this claim here,
I do want to mention one of the experimental tools that is used to
investigate the existence of, and the principles underpinning,
a system of object indexes which operates
between perception and thought ...
 
[Object indexes are going to come in twice: once as a partial solution to the
interface problem, then again as a model for a further conjecture about how it might be solved
(the ‘action index’ conjecture).]
 
\subsection{slide-124}
Suppose you are shown a display involving eight stationary circles, like
this one.
 
\subsection{slide-125}
Four of these circles flash, indicating that you should track these circles.
 
\subsection{slide-126}
All eight circles now begin to move around rapidly, and keep moving unpredictably for some time.
 
\subsection{slide-127}
Then they stop and one of the circles flashes.
Your task is to say whether the flashing circle is one you were supposed to track.
Adults are good at this task \citep{pylyshyn:1988_tracking}, indicating that they can use at least four object indexes simultaneously.
 
They can also sometimes specify which direction the object was moving in (*ref).
 
(\emph{Aside.} That this experiment provides evidence for the existence of
a system of object indexes has been challenged.
See \citet[p.\ 59]{scholl:2009_what}:
\begin{quote}
`I suggest that what Pylyshyn’s (2004) experiments show is exactly what they intuitively
seem to show: We can keep track of the targets in MOT, but not which one is which.
[...]
all of this seems easily explained [...] by the view
that MOT is simply realized by split object-based attention to the MOT targets as a set.'
\end{quote}
It is surely right that the existence of MOT does not, all by itself,
provide support for the existence of a system of object indexes.
However, contra what Scholl seems to be suggesting here, the MOT paradigm
can be adapated to provide such evidence.
Thus, for instance, \citet{horowitz:2010_direction} show that, in a MOT paradigm, observers
can report the direction of one or two targets without advance knowledge of which
targets' directions they will be asked to report.)
 
\subsection{slide-128}
Object indexes are belief-independent.
In this scenario,
a patterned square disappears behind the barrier; later a plain black ring emerges.
If you consider speed and direction only, these movements are consistent with there being just one object.
But given the distinct shapes and textures of these things, it seems all but certain that there must be two objects.
Yet in many cases these two objects will be assigned the same object index \citep{flombaum:2006_temporal,mitroff:2007_space}.
 
\subsection{slide-129}
This system of object indexes
does not involve belief or knowledge
and may assign indexes to objects in ways that are inconsistent with
a subject’s beliefs about the identities of objects
\citep[e.g.][]{Mitroff:2004pc, mitroff:2007_space}
 
Object indexes can conflict with beliefs, although they are surely not
entirely independent of beliefs because they can influence how
attention is allocated.
 
\subsection{slide-130}
Object indexes can guide ongoing object-directed actions.
For example, if you are aiming to catch a moving target,
object indexes are supposed to influence where you reach. (*ref!)
 
\subsection{slide-131}
The fact that object indexes may guide action suggests that they
can contribute to solving
the interface problem?
Suppose you are aiming to catch one of
several possible targets. What do you do? You attend to the target.
What’s going on here?
 
\subsection{slide-132}
... my conjecture is that this is what is going on.
\begin{enumerate}
\item Object indexes enable attention to an object,
\item which enables the formation of an intention,
\item which influences attention to objects,
\item which in turn influences assignments of object indexes,
\item thereby guiding action.
\end{enumerate}
 
\subsection{slide-133}
Now I want to suggest that we can think of the way things work in the case of
object indexes as inspiration for a somewhat analogous idea about how motor representations
might contribute to what we experience when someone acts.
 
What do we experience when someone acts?
Perhaps we can answer this question by analogy with the effects of object indexes on experience.
There’s no perceptual modality linked to objects; instead, a system of object indexes appears to
imposes a sort of structure on experience which can influence attention and guide action.
 
By analogy, we might think that the system of motor representations imposes a sort of structure
on experience which can influence attention too.
 
Earlier I suggested that object indexes provide a partial solution to the interface problem
(Object index->attention->intention->attention->object index->action.)
Similarly, if experience of things around you presents possible actions by way of
structuring, what you need to do to act in a certain way is to attend to a particular action
possibility structure.
 
\subsection{slide-134}
‘Action index’ conjecture

Motor representations of outcomes structure 
experiences, imaginings and (prospective) memories

in ways which provide opportunities for attention to actions directed to those outcomes.

Forming intentions concerning an outcome can influence attention to the action,

which can influence the persistence of a motor representation of the outcome.

 
\subsection{slide-135}
There’s one major disanalogy with object indexes.
Object indexes are about things which are actually present, whereas the motor representations
we are interested in specify possible future outcomes.
 
This appears to be an objection because
on the face of it, it seems that there could not be expeirence of future actions
any more than we can experience future events.
 
So, you might object, the idea that motor representations structure experience is ok if you
are merely observing someone act, but it will not help with performing actions.
 
\subsection{slide-136}
I want to answer this objection in two parts.
The first part of the answer is that motor representations may structure not only
experiences but also imaginigs, memories and prospective memories.
 
\subsection{slide-137}
The second part of the answer to the objection is that motor representations of
outcomes may structure our experiences of objects.
The existence of affordances suggests that this is at least possible.
 
So here is the idea:
What is experienced is an object, not an action or an outcome.
So the fact that the action lies in the future and the outcome has not yet occurred is no objection.
But the motor representation of the outcome structures the experience of the object in some way.
So the overall character of the experience of the mug differs when an action is represented
motorically compared to when it does not.
 
And this difference is structural in the same sense that the difference object indexes make
is structural. It’s about how elements of experience are organised rather than about any
particular sensory modality.
 
\subsection{slide-138}
A second objection: intentions and motor representations need to match in situations where
you merely imagine acting or merely imaginary objects. In such situations there are no objects
to experience.
 
Reply: motor representations structure experiences associated with imaginig things as they do
experiences associated with actually perceiving things. To imagine acting on a mug (say), you need
to imagine the mug.
 
A third objection: close your eyes, put yourself in a sensory deprivation chamber.
Let your hand rest palm down on a table.
Now intend to turn your hand palm up.
Often enough, the intention will succeed.
But by hypothesis there is nothing you experience (you are in a sensory deprivation chamber).
 
Two points in reply to this objection: first, we haven’t removed proprioception and other
somasomatic senses. It may be that motor representations structure experiences of the body
just as much as they structure experiences of mere objects.
 
But what if you remove somasomatic senses too? This is likely to impair action, but unlikely
to make it impossible. Perhaps the ability to act in such situations depends on memory and
imagination.
 
\subsection{slide-139}
So far I’ve suggested that this conjecture (a) might contribute to solving the interface
problem and (b) isn’t obviously wrong. But how could we tell whether it is right?
What predictions does it generate?
 
\subsection{slide-140}
Prediction 1: it is possible to vary which action someone experiences while holding fixed her
perceptual experiences of bodily configurations and joint displacements and their sensory effects.
 
\subsection{slide-141}
 
\subsection{slide-142}
In conclusion,
the interface problem is the problem of explaining how there could be nonaccidental matches
between motor representations and intentions.
 
The interface problem is complicated by the \textbf{anatomy} of outcomes, the varying \textbf{scales}
at which intentions can specify actions, and by the possibility of nonaccidental \textbf{dynamical}
matching.
 
Part (but not all) of the solution to the interface problem may be what I provisionally called
the ‘action index’ conjecture.
According to this conjecture,
‘Action index’ conjecture

Motor representations of outcomes structure 
experiences, imaginings and (prospective) memories

in ways which provide opportunities for attention to actions directed to those outcomes.

Forming intentions concerning an outcome can influence attention to the action,

which can influence the persistence of a motor representation of the outcome.

 
I’m yet fully convinced that this ‘action index’ conjecture is the right way to think about
action experiences. But what I am convinced of is that if we are going to solve the interface
problem, we will need to understand the ways in which actions can be experienced and the mechanisms
which make this possible.
 
%--- end paste
%---------------





\bibliography{$HOME/endnote/phd_biblio}



\end{document}
