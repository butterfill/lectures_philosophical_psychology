 %!TEX TS-program = xelatex
%!TEX encoding = UTF-8 Unicode

%\def \papersize {a5paper}
\def \papersize {a4paper}
%\def \papersize {letterpaper}

%\documentclass[14pt,\papersize]{extarticle}
\documentclass[12pt,\papersize]{extarticle}
% extarticle is like article but can handle 8pt, 9pt, 10pt, 11pt, 12pt, 14pt, 17pt, and 20pt text

\def \ititle {Origins of Mind: Lecture Notes}
\def \isubtitle {Lecture 01}
%comment some of the following out depending on whether anonymous
\def \iauthor {Stephen A.\ Butterfill}
\def \iemail{s.butterfill@warwick.ac.uk% \& corrado.sinigaglia@unimi.it
}
%\def \iauthor {}
%\def \iemail{}
%\date{}

%\input{$HOME/Documents/submissions/preamble_steve_paper4}
\input{$HOME/Documents/submissions/preamble_steve_lecture_notes}

%no indent, space between paragraphs
\usepackage{parskip}

%comment these out if not anonymous:
%\author{}
%\date{}

%for e reader version: small margins
% (remove all for paper!)
%\geometry{headsep=2em} %keep running header away from text
%\geometry{footskip=1.5cm} %keep page numbers away from text
%\geometry{top=1cm} %increase to 3.5 if use header
%\geometry{bottom=2cm} %increase to 3.5 if use header
%\geometry{left=1cm} %increase to 3.5 if use header
%\geometry{right=1cm} %increase to 3.5 if use header

% disables chapter, section and subsection numbering
\setcounter{secnumdepth}{-1}

%avoid overhang
\tolerance=5000

%\setromanfont[Mapping=tex-text]{Sabon LT Std}


%for putting citations into main text (for reading):
% use bibentry command
% nb this doesn’t work with mynewapa style; use apalike for \bibliographystyle
% nb2: use \nobibliography to introduce the readings
\usepackage{bibentry}

%screws up word count for some reason:
%\bibliographystyle{$HOME/Documents/submissions/mynewapa}
\bibliographystyle{apalike}


\begin{document}



\setlength\footnotesep{1em}






%---------------
%--- start paste



\title {Philosophical Psychology \\ Lecture 02: Intention and Motor Representation in Purposive Action}



\maketitle

\subsection{title-slide}
Any attempt to bring scientific discoveries about action into a philosophical
discussion quickly runs into a significant obstacle ...
[The obstacle will be apparently completely different way of thinking about action.]

\subsection{slide-3}
Which events are actions?
In philosophy, answering this question would typically answered by appeal to intention
or practical reasoning.

Such views tend to be neutral on how
the attitudes and processes ultimately connect to bodily movements;
that is considered to be merely an implementation detail ...

They are neutral in this sense: the views do not depend in any way on facts about that
distinguish one kind of body from another, or on facts about how the body’s movements
are ultimately controlled ...

\subsection{slide-4}
In cognitive science ... little to say about actions whose purposes
involve things the motor system doesn’t care about---your motor system
doesn’t care whether the plane you are stepping is headed for Milan or
for Rome, but this sort of difference can affect whether your actions succeed or fail.

\subsection{slide-5}
You might just say that the two disciplines are talking past each other,
or you might say that they are offering two complementary but independent
models of action.

Call this the ‘Two Stories View’ (or divorced, but living together).

But we want to argue that these two views are components of a single, larger story about action.
Although intention and motor representation can usefully be studied in isolation
to some extent, a full understanding of action will require understanding
interfaces between the two …

But first let me fill in a little detail about each of the stories ...

\subsection{intro\_to\_intention}


\section{Intention in the Philosophy of Action}

What are intentions and what is their role in practical reasoning and action? Some basic
distinctions from the Philosophy of Action can give us a fix on the notion of intention as
many philosophers conceive of it.

\subsection{slide-18}
nothing wrong with desiring these things ... is something wrong with intending them
Agglomeration is important because it distinguishes intention from desire, and from strongest desire.

\subsection{slide-19}
So what are intentions for?

I’m going to assume that intention is something over and above basic beliefs and desires; that an
intention is not, for instance, merely a strongest desire.

There is a temptation to assume that intention is involved in every case of purposive action.
But it’s hard to see what the argument for this could be.
In many cases it seems that beliefs, desires and motor representations are all that is needed to explain purposive action.
You offer me a biscuit.  I want one, and I believe I can get one by reaching out for it.  So I do reach for it.  As far as I can see, there’s no need to suppose that, in addition to the belief and desire, it must be the case that I also intend to take a biscuit.
(At least not unless we take ‘intention’ to mean ‘strongest desire’, which it does not.)
Maybe I do intend this.
But it’s possible for an agent to take and eat a biscuit, and to do so purposively, without having any intentions at all.
Beliefs desires and motor representations are sufficient.

So (again), what are intentions for?
The answer has something to do with planning.
But here we must be careful because not all planning involves intending.

\subsection{slide-21}
As this illustrates,
some actions involving are purposive in the sense that

\subsection{slide-22}
among all their actual and possible consequences,

\subsection{slide-23}
there are outcomes to which they are directed

\subsection{slide-24}
and the actions are collectively directed to this outcome

\subsection{slide-25}
so it is not just a matter of each individual action being directed to this outcome.

\subsection{slide-26}
In such cases we can say that the actions are clearly purposive.

\subsection{slide-27}
Concerning any such actions, we can ask
What is the relation between a purposive action and the outcome or outcomes to which it is directed?

\subsection{slide-28}
The standard answer to this question involves intention.

\subsection{slide-29}
An intention (1) specifies an outcome,

\subsection{slide-30}
(2) coordinates the one or several activities which comprise the action;

and (3) coordinate these activities in a way that would normally facilitate the outcome’s occurrence.

What binds particular component actions together into larger purposive actions?
It is the fact that these actions are all parts of plans involving a single intention.
What singles out an actual or possible outcome as one to which the component
actions are collectively directed?  It is the fact that this outcome is
represented by the intention.

So the intention is what binds component actions together into purposive actions and
links the action taken as a whole to the outcomes to which they are directed.

\subsection{slide-31}
But is intention the only thing that can link actions to outcomes?
I will suggest that motor representations can likewise perform this role.

\subsection{intro\_to\_motor\_representation}


\section{Motor Representation}

Motor representations are involved in performing and preparing actions.
Not all representations represent patterns of joint displacements and bodily
configurations: some represent outcomes such as the grasping of an object, which
may done in different ways in different contexts.

\subsection{slide-33}
Let me mention some almost uncontroversial facts about motor representations and
their action-coordinating role.

\subsection{slide-34}
Suppose you are a cook who needs to take an egg from its box, crack it and put it (except for the
shell) into a bowl ready for beating into a carbonara sauce.
Even for such mundane, routine actions, the constraints on adequate performance can vary
significantly depending on subtle variations in context. For example, the position of a hot pan
may require altering the trajectory along which the egg is transported, or time pressures may mean
that the action must be performed unusually swiftly on this occasion.
Further, many of the constraints on performance involve relations between actions occurring at
different times.
To illustrate, how tightly you need to grip the egg now depends, among other things, on the forces
to which you will subject the egg in lifting it later.
It turns out that people reliably grip objects such as eggs just tightly enough across a range of
conditions in which the optimal tightness of grip varies.
This indicates (along with much other evidence) that information about the cook’s anticipated
future hand and arm movements appropriately influences how tightly she initially grips the egg
(compare \citealp{kawato:1999_internal}).
This anticipatory control of grasp,
like several other features of action performance (\citealp[see][chapter 1]{rosenbaum:2010_human} for more examples),
is not plausibly a consequence of mindless physiology, nor of intention and practical reasoning.
This is one reason for postulating motor representations, which characteristically play a role in
coordinating sequences of very small scale actions such as grasping an egg in order to lift it.

The scale of an actual action can be defined in terms of means-end relations.
Given two actions which are related as means to ends by the processes and representations
involved in their performance, the first is smaller in scale than the second just if the
first is a means to the second.  Generalising, we associate the scale of an actual action
with the depth of the hierarchy of outcomes that are related to it by the transitive closure
of the means-ends relation.  Then, generalising further, a repeatable action (something that
different agents might do independently on several occasions) is associated with a scale
characteristic of the things people do when they perform that action.  Given that actions
such as cooking a meal or painting a house count as small-scale actions, actions such as
grasping an egg or dipping a brush into a can of paint are very-small scale.  Note that we
do not stipulate a tight link between the very small scale and the motoric.  In some cases
intentions may play a role in coordinating sequences of very small scale purposive actions,
and in some cases motor representations may concern actions which are not very small scale.
The claim we wish to consider is only that, often enough, explaining the coordination of
sequences of very small scale actions appears to involve representations but not, or not
only, intentions.  To a first approximation, \emph{motor representation} is a label for
such representations.%
\footnote{%
Much more to be said about what motor representations are; for instance, see \citet{butterfill:2012_intention} for the view that motor representations can be distinguished by representational format.
}

\subsection{slide-35}
What do motor representations represent? An initially attractive, conservative
view would be that they represent bodily configurations and joint displacements,
or perhaps sequences of these, only.
However there is now a significant body of evidence that some motor representations
do not specify particular sequences of bodily configurations and joint displacements,
but rather represent outcomes such as the grasping of an egg or the pressing of a switch.
These are outcomes which might, on different occasions, involve very different bodily
configurations and joint displacements
(see \citealp{rizzolatti_functional_2010} for a selective review).

Such outcomes are abstract relative to bodily configurations and joint displacements
in that there are many different ways of achieving them.

But how do we know that motor representations carry information about such outcomes?
I’m glad you asked, let me explain ...

\subsection{slide-36}
If you were to observe someone phi-ing,
then motor representations would occur in you
much like those that would occur in you if it were you, not her, who was phi-ing.

This will be a focus of interest in a later session.
For now it’s just a handy fact that simplifies testing.

\subsection{slide-39}
‘The posterior section of the frontal lobe contains the motor areas,’ (p.~4)
Now you know as much about the brain as I do.

Mention primary and supplementary motor areas : we use the term ‘motor’ loosely
(compare ‘visual’, which also has narrower and broader uses in
neuroscience).

\subsection{slide-40}
TMS to measure MEP

\subsection{slide-41}
They also had an occluded end version ...

\subsection{slide-42}
Incidentally, ‘the observed direction of the modulation was not consistent with previous TMS
literature. Specifically, MEP amplitudes were significantly lower in the Object-Present than in the
Object-Absent conditions (Fig. 2), suggesting that there was an inhibitory effect of object
manipulation on the activity of M1 during action observation.’

\subsection{slide-46}
Umiltà et al, 2008 : single cell recordings in monkeys

MEPs (TMS amplified) in humans

\subsection{slide-47}
TMS MEP, humans.

Shown video, then a static picture.
Is this the same goal as you saw in the video?
Press one of two keys.
‘They were explicitly told to ignore the effector and make a judgment on the type of act only.’

\subsection{slide-48}
Key finding: TMS to both ventral premotor cortex (PMv) and left supramarginal gyrus (SMG)
increases RTs regardless of whether it’s the same effector or a different effector.
(You can’t see same/different effector in this figure.)

KEY: superior temporal sulcus (STS), and a parietofrontal system consisting of the intraparietal
sulcus (IPS) and inferior parietal lobule (IPL) plus the ventral premotor cortex (PMv) and caudal
part of inferior frontal gyrus (IFG). In some instances also, the superior parietal lobule (SPL)

\subsection{slide-49}
By contrast, TMS to superior temporal sulcus (STS) increased RT only for judgements
where the video effector was the same as the photo effector.

\subsection{slide-50}
The experiments providing such evidence typically involve a marker of motor representation,
such as a pattern of neuronal firings, a motor evoked potential or a behavioural performance
profile, which, in controlled settings, allows sameness or difference of motor representation
to be distinguished.  Such markers can be exploited to show that the sameness and difference
of motor representation is linked to the sameness and difference of an outcome such as the
grasping of a particular object.
(Pioneering uses of this method include \citealp{rizzolatti:1988_functional,Rizzolatti:2001ug};
it has since been developed in many ways: see, for example,
\citet{hamilton:2008_action, cattaneo:2009_representation, cattaneo:2010_state-dependent,
rochat:2010_responses, bonini:2010_ventral, koch:2010_resonance}.)

\subsection{slide-58}
To illustrate, consider a sequence of actions which might be involved in shoplifting an apple: you have to secure the apple, transport it, and position it in your pocket.
Each of these outcomes can be represented motorically.

\subsection{slide-59}
Motor processes are planning-like in that they involve computing means from ends.
Thus a representation of an end like securing it [the apple] can trigger a process
that results in the representation of outcomes that are means to this end.

\subsection{slide-60}
Motor processes are also planning-like in that which means are selected in preparing an
action that will occur early in the sequence may affect needs that will arise only later
in a later part of the actions.
For instance, how the apple is grasped at an early point in the sequence may be determined
in part by what would be a more comfortable way for the other hand to grasp it later.

\subsection{slide-61}
So motor representations of outcomes guide planning-like processes.
This is why I think it’s not just that they carry information about outcomes
like grasping an apply, but that they also represent such outcomes.

\subsection{motor\_representations\_ground\_goals}


\section{Motor Representations Ground the Directedness of Actions to Goals}

How do intentions ground the purposiveness of actions? On any standard view, an intention
represents an outcome, causes an action, and does so in a way that would normally facilitate
the outcome’s occurrence. Similarly, some motor representations represent action
outcomes, play a role in generating actions, and do this in a way that normally facilitates
the occurrence of the outcomes represented.  Like intentions, motor representations
ground the directedness of actions to outcomes which are thereby goals of the actions.

\subsection{slide-64}

\subsection{slide-65}
Now as Elisabeth Pacherie has argued (and I’ve had a go at arguing this in joint work with Corrado Sinigaglia recently too),
motor representations are relevantly similar to intentions.

Of course motor representations differ from intentions in some important ways (as Pacherie also notes).

But they are similar in the respects that matter for explaining the purposiveness of action.
(1) Like intentions, some motor representations represent outcomes (and not merely patters of joint displacement, say).
(2) Like intentions, some motor representations play a role in coordinating multiple more component activities by virtue of their role as elements in hierarchically structured plans.
(3) And, like intentions, some motor representations coordinate these activities in a way that would normally facilitate the outcome’s occurrence.

The claim is not that \emph{all} purposive actions are linked to outcomes by motor representations, just that some are.
In some cases, the purposiveness of an action is grounded in a motor representation of an outcome; in other cases it is grounded in an intention.

And of course in many cases it may be that both intention and motor representation are involved.

\subsection{motor\_representations\_arent\_intentions}


\section{Motor Representations Aren’t Intentions}

Explains why motor representations aren’t intentions.

\subsection{slide-69}
As background we first need a generic distinction between content and format.
Imagine you are in an unfamiliar city and are trying to get to the central station.
A stranger offers you two routes. Each route could be represented by a distinct line
on a paper map. The difference between the two lines is a difference in content.

\subsection{slide-70}
Each of the routes could alternatively have been represented by a distinct series
of instructions written on the same piece of paper; these cartographic and
propositional representations differ in format. The format of a representation
constrains its possible contents. For example, a representation with a cartographic
format cannot represent what is represented by sentences such as `There could not be a
mountain whose summit is inaccessible.'\footnote{ Note that the distinction between
content and format is orthogonal to issues about representational medium. The maps in
our illustration may be paper map or electronic maps, and the instructions may be spoken,
signed or written. This difference is one of medium.} The distinction between content and
format is necessary because, as our illustration shows, each can be varied independently
of the other.

\subsection{slide-71}
Format matters because only where two representations have the same format can they be straightforwardly inferentially integrated.

To illustrate, let’s stay with representations of routes.
Suppose you are given some verbal instructions describing a route. You are then shown a representation of a route on a map and asked whether this is the same route that was verbally described. You are not allowed to find out by following the routes or by imagining following them.
Special cases aside, answering the question will involve a process of translation because two distinct representational formats are involved, propositional and cartographic. It is not be enough that you could follow either representation of the route. You will also need to be able to translate from at least one representational format into at least one other format.

\subsection{slide-72}
How in general can we identify or distinguish representational formats? Because representational formats are typically associated with characteristic performance profiles, it is sometimes possible to infer similarities and differences in representational format from similarities and differences in the processes in which representations feature.

To illustrate, suppose that you have a route representation and I want to work out whether it this representation has a cartographic or propositional format.  One way to do this might be to test your performance on different tasks.  If the representation is propositional you are likely to be relatively fast at identifying key landmarks but relatively slow at translating the route into a sequence of compass directions; but the converse will be true if your representation is cartographic.

\subsection{slide-73}
The same principle---distinguishing and identifying formats by measuring characteristic processing profile---works for mental representations too.

To illustrate, compare imagining seeing an object moving with actually seeing it move.
For this comparison we need to distinguish two ways of imagining seeing. There is a way of
imagining seeing which phenomenologically is something like seeing except that it does not
necessarily involve being receptive to stimuli. This way of imagining seeing, sometimes
called `sensory imagining', is commonly distinguished from cognitive ways of imagining
seeing which might for example involve thinking about seeing.
It is this way of imagining seeing an object move that we wish to compare with actually
seeing an object move.

\subsection{slide-74}
Imagining seeing an object move and actually seeing an object move have similarities in characteristic performance profile.  For instance, whether an object can be seen all at once depends on its size and distance from the perceiver; strikingly, when subjects imagine seeing an object, whether they can imagine seeing it all at once depends in the same way on size and distance (\citealp{kosslyn:1978_measuring}; \citealp[p.\ 99ff]{kosslyn:1994_image}).

Also, how long it takes to imagine looking over an object depends on the object's subjective size in the same way that how long it would take to actually look over that object would depend on its subjective size \citep{kosslyn:1978_visual}.

The similarities in characteristic performance profile and the particular patterns of interference are good (if non-decisive) reasons to conjecture that imagining seeing and actually seeing involve representations with a common format.

\subsection{slide-75}
One way of imagining action is phenomenologically something like acting except
that such imaginings are not necessarily responsive to the features of actual
objects and do not necessarily result in bodily movements.

There is evidence that the way imagining performing an action unfolds in time is
similar in some respects to the way actually performing an action of the same type would unfold.

For instance, how long it takes to imagine moving an object is closely related to
how long it would take to actually move that object \citep{decety:1989_timing,
decety:1996_imagined, Jeannerod:1994oz}.

In addition, for actions such as grasping the handle of a cup, manipulating the
target object in ways that would make the action harder (such as orienting the cup's
handle to make it less convenient for you to grasp) make a corresponding difference to
the effort involved in imagining performing the action \citep{parsons:1994_temporal,
frak:2001_orientation}.

\subsection{slide-76}
Contrast imagining rotating a ball with imagining seeing a ball rotate.

As is implied by what we’ve already said, these have quite different characteristic performance profiles.

How quickly the former can be done is a function of how long it would take the agent to rotate the ball, whereas how quickly the latter can be done depends on how rapidly the ball can rotate and still be perceived as rotating.

Further, in some cases rotating a ball clockwise is easier than rotating it anti-clockwise, and so is imagining a ball rotate.  By contrast, the effort involved in actually seeing or imagining seeing a ball rotate does not similarly differ depending on direction.

\subsection{slide-77}

\subsection{slide-78}

\subsection{slide-83}
So where does this leave us with respect to our starting point,
the ‘Two Stories’ view?

On the face of it, everything I’ve said so far is compatible with that View
and might even be taken to support it.

But there is a problem ...

\subsection{slide-84}
Recall stealing an apple ...

\subsection{slide-85}
Imagine you are a hungry shoplifter with a powerful desire for an apple.  After carefully
considering the pile of apples on the stall, you identify one that can be discretely
snatched and form an intention to steal that apple as you casually saunter past the stall.
In grasping, transporting and pocketing the apple your movements are controlled by motor
representations of these outcomes.  So your theft depends on an intention, on some motor
representations and on there being a match between the outcome specified by the intention
and the outcomes specified by the motor representations.

In short: practical reasoning and motor processes are part of a single story about the
performance of action; there must be non-accidental matches between the contents of
intentions and motor representations.

\textbf{So we can’t accept the Two Stories View because there must sometimes be content-respecting
causal interactions involving intentions and motor representations.
Without these, our intentions and motor representations would never non-accidentally
have synergistic effects on our actions.}

\subsection{the\_interface\_problem}


\section{The Interface Problem}

For a single action, which outcomes it is directed to may be multiply
determined by an intention and, seemingly independently, by a motor
representation. Unless such intentions and motor representations are to pull
an agent in incompatible directions, which would typically impair action
execution, there are requirements concerning how the outcomes they represent
must be related to each other. This is the interface problem: explain how any
such requirements could be non-accidentally met.

\subsection{slide-92}
I want to offer a conjecture about how the interface problem is solved ...

\subsection{slide-94}
Having an intention is neither necessary nor sufficient.

\subsection{slide-95}
Jeannerod 2006, p. 12:
‘the term apraxia was coined by Liepmann to account for higher order motor
disorders observed in patients who, in spite of having no problem in executing
simple actions (e.g. grasping an object), fail in actions involving more
complex, and perhaps more conceptual, representations.’

Can people with ideomotor apraxia form intentions?

\subsection{slide-96}
\citep[p.~7]{mylopoulos:2016_intentions}: ‘Typically the result of lesion
to SMA or anterior corpus callosum, AHS is a condition in which patients
perform complex, goal-oriented movements with their cross-lesional limb
that they feel unable to directly inhibit or control. The limb is often
disproportionately reactive to environ- mental stimuli, carrying out
habitual behaviors that are inappropriate to the context, e.g., grabbing
food from a dinner companion’s plate (Della Sala 2005, 606). It is clear
from many of the behaviors observed in these cases that the anarchic limb
fails to hook up with the agent’s intentions. ’

\subsection{slide-97}
It’s no mystery that their are nonaccidental matches between your desires and your
intentions. After all, these are linked by a process of practical reasoning. One of the
key jobs of practical reasoning is to ensure that, within limits, your intentions match
your desires.

Why not suppose that some related, inference-like process links your intentions
to your motor representations?
So you would form an intention, and then some process analogous to practical reasoning
would be engaged to ensure that matching motor representations occur in you.

\subsection{slide-98}
The main obstacle is the difference in representational format.
This means that inferrential integration, or any kind of content-respecting
causal process would require translation between representational formats.
And as far as we know there is no such process of translation.

\subsection{slide-99}
Mylopoulos and Pacherie claim to offer
‘a version of the content-respecting causal
processes solution’

Let’s consider their view ...

\subsection{slide-100}
Mylopoulos and Pacherie’s idea has two parts, motor schema and executable action concept.

\subsection{slide-101}
Mylopoulos and Pacherie’s ‘motor schema’ idea.

‘Motor schemas are more abstract and stable representations of actions than motor
representations.’

‘They are internal models or stored representations that represent generic knowledge
about a certain pattern of action and are implicated in the production and control of
action. For instance, in the influential Motor Schema Theory proposed by Richard Schmidt
(1975, 2003), a motor schema involves a generalized motor program, together with corre-
sponding ‘recall’ and ‘recognition’ schemas. The generalized motor program is thought to
contain an abstract representation defining the general form or pattern of an action,
that is the organization and structure common to a set of motor acts (e.g., invariant
features pertaining to the order of events, their spatial configuration, their relative
timing and the relative force with which they are produced). This generalized motor
program has parameters that control it. In order to determine how an action should be
performed on a given occasion, parameter values adapted to the situation must be
specified. Thus, a motor schema also includes a rule or system of rules describing the
relationships between initial conditions, parameter values and outcomes and allowing us
to perform the action over a large range of conditions (the ‘recall schema’ in Schmidt’s
terminology). Finally, the motor schema also includes a rule or system of rules
describing the relationships between initial conditions, exteroceptive and propri-
oceptive sensory feedback during an action, and action outcome (a recognition sche- ma),
allowing agents to know when they have made an error – i.e., the action does not have
the sensory consequences it is expected to have – and to correct for it.’

Are motor schemas an alterantive to motor representations? Tempting to think that where
they talk about motor schema, we think of motor representations that represent outcomes
that are relatively distal from action (e.g. are affector-neutral).

This would explain why they contrast motor schema with motor programmes. We use ‘motor
representation’ to include both.

Consider an example ...

\subsection{slide-102}
In Mylopoulos and Pacherie’s terms, you would need to say that this involves sameness of schema rather than sameness of motor representation.

I think the difference is mainly terminological.  Because we are uncertain about
the nature of the motoric, we don’t distinguish motor representations that
specify actions in more detail (particular effector, particular grip) from
motor representations that specify actions in less detail.
Because they think that motor representations (or motor programmes) must be tied to
particular actions, they need motor schema.
(Partly this debate may hinge on the importance of bodily synergies and
nonrepresentational processes in the control of action: how much has to be represented
at all? More than intentions. But not necessarily each joint displacement. So we don’t
currently know where representation ends.)

\subsection{slide-104}
There explanation: two different kinds of action concept, one purely
observational.
‘For instance, most of us have a concept of ‘tail wagging’ that we can deploy when we
judge, for instance, that Julius the dog is wagging his tail. Or if you are not
convinced that tail wagging constitutes purposive behavior, consider the action concept
‘tail swinging’, as in ‘cows constantly swing their tails to flick away flies’.’

A demonstrative action concept would be an executable action concept,
but not convesely.
So this is potentially a good generalisation.

But regardless of this, how on their view do the motor schema connect to the
executable action concepts?

\subsection{slide-105}
What you represent motorically is not only the actions you will perform.
Many potential actions are represented motorically, depending on what a situation
affords you.
Only some of the affordances represented become actions.

What we want is this: what you intend influences
which motor representations lead to action.

To echo Velleman, we want intentions to throw their weight behind some
motor representations and not others.

How could this happen?

I think part of the idea of an executable action concept is this.
The concept is associated with a motor representation (or schema),
and so activating the concept strengthens the associated motor
representation.

So far, so good.  There are just two problems ...

\subsection{slide-106}
‘As defined by Tutiya et al., an executable concept of a type of movement is a
representation, that could guide the formation of a volition, itself the proximal cause of
a corresponding movement. Possession of an executable concept of a type of movement thus
implies a capacity to form volitions that cause the production of movements that are
instances of that type.’
\citep[p.~7]{pacherie:2011_nonconceptual}

It looks like the executable action concept builds in a solution to the interface problem
rather than solving it.
The executable action concept is a ready-made solution.
This is the first problem.

\subsection{slide-107}
The second problem is based on a general assumption.
The assumption is that in using a concept, you know which thing you are thinking about.

A concept is not a brute way of relating a thinker to a thing; it is a way that provides
the thinker with some kind of cognitive connection to the thing in question.

That is why having a concept cannot be a matter of merely having a mental symbol
which is associated in some way with the thing it is a concept of.
(Compare the corresponding view about words.)

How does using an executable action concept give me insight into which action I’m thinking
about?

\subsection{slide-108}
‘Motor schemas are thus, we submit, what bridges the gap between intentions and motor representations, ensuring proper, content-preserving coordination without requiring any mysterious translation process.’

This is good.  But how do you connect executable action concepts to motor schema?

M\&P say that there are concepts tied to motor schema, and provide some evidence
for this.  I think this is progress.
Executable action concepts are a solution to the interface problem. We know they exist.
They probably don’t always involve demonstratives that refer to actions by deferring to
motor representations.
So we should step back.

\subsection{slide-110}
Mylopoulos and Pacherie claim to offer ‘a version of the content-respecting causal
processes solution’

But is their view really a content-respecting causal processes solution?
As far as I can see, there is no reason to assume that it is.
Everything depends on how executive action concepts get hooked up to
the motoric.

And this could be a matter of concepts and motor schema (or representations) somehow
becoming associated.


\subsection{slide-111}
How do you connect a concept to a motor representation (or schema)?
This looks like a version of the interface problem ...

\subsection{slide-112}
There is a way to make the problem of comparison between representational formats trivial

\subsection{slide-113}
Suppose one representation involves a demonstrative that refers by
deferring to another representation

\subsection{slide-114}
Then the comparison doesn’t require translation between formats after all.
Maybe the same can be true for intentions and motor representations.
Maybe intentions can involve demonstrative concepts which refer to actions by deferring to motor representations?

[*cut:
Set that aside, suppose it can be solved --- essentially because MR must give rise to
experience of action.
On this view, it is demonstrative deference to motor representation that connects
intentions to bodily movements.
Only by recognising how intentions interlock with motor representations can we hope to
understand how our intentions ever make a difference to the world around us.
On this view experience of action plays a novel role.
Action experiences in which motor representations feature, such as those associated with
motor imagery and those associated with really acting, are arguably necessary for there to
be concepts which are constituents of intentions and refer to actions by deferring to
motor representations.
But if, as we conjecture, such deference is necessary for intentions to properly and
reliably result in bodily movements, it may turn out that intentionally acting in the
world de- pends on action experiences featuring motor representation.
Much as on some views thought about objects depends on perceptual experience (e.g.
Campbell 2002), so also intending actions may depend on motor experience.

\subsection{slide-116}
There are demonstratives which refer to action types.
(The wish was not to perform another agent’s action but to perform an action of a certain
type.)

\subsection{slide-117}
How are such demonstratives possible? What determines which action is referred to?

\subsection{slide-118}
In general, demonstrative reference depends on experience of the referent (Campbell).
But what sort of experience do we have of the referent?

\subsection{slide-119}
What experience?   One which also occurs in pantomime ...

\subsection{slide-120}
Observation: you can refer demonstratively to an action that you pantomime.

Our example: ‘consider purely mental pantomime—that is, phenomenologically
action-like imagination. One might use this kind of imagination to explore
different ways of completing a task and then, having hit on a good
solution, think to oneself ‘Do that!’. It seems possible that in some such
cases the demonstrative refers by deferring to a motor representation of
action involved in imagining acting (135)’

\subsection{slide-121}
Experiences associated with act types are (i) shaped by motor representations and (ii)
make act types possible objects of demonstrative reference.

\subsection{slide-122}
Our idea, then, is that you can
refer to an action by deferring to a motor representation

-- motor representation shapes experience



-- experience is of action



-- experience enables demonstrative reference to action



-- which action is determined by what is represented motorically (deference).


\subsection{slide-124}
So Mylopoulos \& Pacherie say we can solve the Interface Problem if we have
executable action concepts.  This seems to raise the question,
How do executable action concept hook up to motor representations?
And this seems to be a version of the Interface Problem.

\subsection{slide-125}
But we can solve that problem by supposing that executable action concept
are demonstrative concepts
which refer to actions by deferring to motor representations.
Or so I’ve suggested.

\subsection{slide-128}
Recall the analogy with maps:

\subsection{slide-129}
using a map to
refer demonstratively to a route by deferring to a representation of it
solves an interface problem.

\subsection{slide-130}
But we can’t point to motor representations like we can point to maps!
Mylopoulous \& Pacherie make this vivid ...

\subsection{slide-131}
... Mylopoulous \& Pacherie (2016, 11) when they write that

‘in the case of intention, there is no such perceptual-attentional link with the motor representation; these are not mental representations that we can attend to or perceive.’

\subsection{slide-132}
But we can’t point to motor representations like we can point to maps!

What we need for reference by deference to a motor representation is
experience of motor action.

And if you think about motor imagination it seems quite plausible that we
do have such experiences.

So here’s the thought:
There are no direct inferential connections between intentions and motor
representations. Harmony is ensured by the fact that where an intention
involves a bodily movement, either executing that intention involves
forming a further intention or else the intention involves a demonstrative
that refers to an action by deferring to a motor representation.

So what connects intentions to motor representations---what connects the reflective to the
pre-reflective---is the use of demonstratives, and this depends on experience of motor
action.

Much as on some views all thought about objects ultimately depends on
perceptual experience (e.g. Campbell 2002), so also intending bodily
actions may ultimately depend on motor experience. Experience anchors the
reflective in the pre-reflective.

\subsection{slide-133}
Our view requires that motor representations enable consciousness of the
action referred to.

\subsection{slide-134}
Mylopoulos and Pacherie object (p. 11): ‘there are no clear examples of intention deferring
to motor representation, which makes it dubious that this ever takes place’

Our example was: ‘consider purely mental pantomime—that is, phenomenologically
action-like imagination. One might use this kind of imagination to explore
different ways of completing a task and then, having hit on a good
solution, think to oneself ‘Do that!’. It seems possible that in some such
cases the demonstrative refers by deferring to a motor representation of
action involved in imagining acting (135)’

Mylopoulos and Pacherie reply (p. 11): ‘The obvious problem here is that we need not
suppose that the intention refers to the relevant action by way of deferring to a motor
representation. What seems more likely is that it does so by deferring to the mental image
of the relevant action.’

But what is a mental image of an action? Maybe it’s nothing other than the
experience that the motor representation gives rise to.  If so, there is no
real disagreement here.

\subsection{slide-135}
‘Yet another pressing issue for Butterfill and Sinigaglia’s view is that it seems not to
have the resources to explain errors in action execution.’

\subsection{slide-136}
No problem here since intentions are not the only cause of motor representations ...

\subsection{slide-137}
Example: you intend to put on your shirt and this very intention causes you
to comb your hair.  (*Better example?)

Mylopoulos and Pacherie are assuming that this requires defernce to the
‘wrong’ motor representation.
We agree that this can’t happen --- we can’t easily make sense of the possibility that
your failed to refer, in intention, to the action you were trying to pick out.

\subsection{slide-138}
‘In the case of a regular demonstrative utterance, the speaker has an independent way of picking out which object to refer to. If I want to demonstratively refer to the green apple in the heap, I will employ my perceptual resources in order to determine which item to point to. Similarly, in the case of demonstrative deferral in intention, the agent must have an independent grasp of which motor representation is the appropriate one to select via such deferral’

I don’t understand this objection.  One problem is that the analogy seems to require
only that we have an independent way of picking out the thing referred to.
But this is the action, not the motor representation.
(It is not part of our view that we demonstratively refer to motor representations.)

\subsection{slide-140}
The analogy we’d go for is: I will employ my motor resources in order to determine
which action to perform.





%--- end paste
%---------------





\bibliography{$HOME/endnote/phd_biblio}



\end{document}
