%!TEX TS-program = xelatex
%!TEX encoding = UTF-8 Unicode

\documentclass[12pt]{extarticle}
% extarticle is like article but can handle 8pt, 9pt, 10pt, 11pt, 12pt, 14pt, 17pt, and 20pt text

\def \ititle {Origins of Mind}
 
\def \isubtitle {Lecture 01}
 
\def \iauthor {Stephen A. Butterfill}
\def \iemail{s.butterfill@warwick.ac.uk}
\date{}

%for strikethrough
\usepackage[normalem]{ulem}

\input{$HOME/Documents/submissions/preamble_steve_handout}

%\bibpunct{}{}{,}{s}{}{,}  %use superscript TICS style bib
%remove hanging indent for TICS style bib
%TODO doesnt work
\setlength{\bibhang}{0em}
%\setlength{\bibsep}{0.5em}


%itemize bullet should be dash
\renewcommand{\labelitemi}{$-$}

\begin{document}

\begin{multicols*}{3}

\setlength\footnotesep{1em}


\bibliographystyle{newapa} %apalike

%\maketitle
%\tableofcontents




%--------------- 
%--- start paste

      

    
      
\def \ititle {Joint Action}
 
\def \isubtitle {Lecture 01}
 
\begin{center}
 
{\Large
 
\textbf{\ititle}: \isubtitle
 
}
 
 
 
\iemail %
 
\end{center}
 
 
 
\section{The Question}
 
Which forms of shared agency underpin our social nature?

 
A \emph{joint action} is an exercise of shared agency.
 
Modest sociality: ‘small scale shared intentional agency in the absence of 
asymmetric authority relations’
\citep[p.~150]{Bratman:2009lv}
 
 
 
\section{Contrast Cases and the Simple View}
 
Question


            
What distinguishes genuine joint actions from parallel but merely individual actions?
            

 
Aim


              
An account of joint action must draw a line between joint actions and parallel but 
merely individual actions.
              

 
\emph{The Simple View}
 
Two or more agents perform an intentional joint action exactly when there is an act-type, φ, such that each of several agents intends that they, these agents, φ  together and their intentions are  appropriately related  to their actions.

 
 
 
\section{Acting Together and the Threat of Circularity}
 
‘Examples of what I shall refer to ... as “acting together” include dancing together, building a house together, and marching together against the enemy, where these are construed as something other than a matter of doing the same thing concurrently and in the same place’ 
\citep[p.~23]{gilbert:2014_book}
 
‘The key question in the philosophy of collective action is simply ... under what conditions are two or more people doing something together?’
\citep[p.\ 67]{Gilbert:2010fk}
 
‘two or more people are acting together if [and only if] they are jointly committed  to espousing as a body a certain goal, and each one is acting in a way appropriate to the achievement of that goal, where each one is doing this in light of the fact that he or she is subject to a joint commitment to espouse the goal in question as a body.’
\citep[p.~34]{gilbert:2014_book}
 
‘any random group of agents is a group that does something together’
\citep[p.~128]{ludwig:2014_ontology}
 
 
 
\section{Walking Together in the Tarantino Sense}
 
`each agent does not just intend that the group perform the […] joint action. Rather, each agent intends as well that the group perform this joint action in accordance with subplans (of the intentions in favor of the joint action) that mesh' \citep[p.\ 332]{Bratman:1992mi}.
 
Our plans are \emph{interconnected} just if facts about your plans feature in mine and conversely.
 
‘shared intentional agency consists, at bottom, in interconnected planning agency of the participants’ \citep{Bratman:2011fk}.
 
\begin{minipage}{\columnwidth}
 
\emph{Bratman’s claim}. For you and I to have a collective/shared intention that we J it is sufficient that:
 
\begin{enumerate}[label=({\arabic*}),itemsep=0pt,topsep=0pt]
 
\item  `(a) I intend that we J and (b) you intend that we J;
 
\item `I intend that we J in accordance with and because of la, lb, and meshing subplans of la and lb; you intend that we J in accordance with and because of la, lb, and meshing subplans of la and lb;
 
\item `1 and 2 are common knowledge between us' \citep[View 4]{Bratman:1993je}
 
\end{enumerate}
 
\end{minipage}
 
\section{Objectives for this lecture}
 \begin{itemize}
\item 
understand questions about shared agency

          
\item can use the method of contrast cases

          
\item understand distributive and collective interpretations of sentences

          
\item can distinguish acting together from joint action

          
\item familiar with the Simple View

          
\item can critically assess objections to the Simple View

\end{itemize}
 

    
%--- end paste
%--------------- 
 
\footnotesize 
\bibliography{$HOME/endnote/phd_biblio}

\end{multicols*}

\end{document}