%!TEX TS-program = xelatex
%!TEX encoding = UTF-8 Unicode

\documentclass[12pt]{extarticle}
% extarticle is like article but can handle 8pt, 9pt, 10pt, 11pt, 12pt, 14pt, 17pt, and 20pt text

\def \ititle {Origins of Mind}

\def \isubtitle {Lecture 01}

\def \iauthor {Stephen A. Butterfill}
\def \iemail{s.butterfill@warwick.ac.uk}
\date{}

%for strikethrough
\usepackage[normalem]{ulem}

\input{$HOME/Documents/submissions/preamble_steve_handout}

%\bibpunct{}{}{,}{s}{}{,}  %use superscript TICS style bib
%remove hanging indent for TICS style bib
%TODO doesnt work
\setlength{\bibhang}{0em}
%\setlength{\bibsep}{0.5em}


%itemize bullet should be dash
\renewcommand{\labelitemi}{$-$}

\begin{document}

\begin{multicols*}{3}

\setlength\footnotesep{1em}


\bibliographystyle{newapa} %apalike

%\maketitle
%\tableofcontents




%---------------
%--- start paste




\def \ititle {Lecture 02: Intention and Motor Representation in Purposive Action}

\begin{center}

{\Large

\textbf{\ititle}

}



\iemail %

\end{center}



\section{Intention in the Philosophy of Action}

‘The expression ‘the intention with which James went to church’ ... cannot be taken to refer to a
... state .... Its function ... is to generate new descriptions of actions in terms of their
reasons; thus ‘James went to church with the intention of pleasing his mother’ yields a new, and
fuller, description of the action described in ‘James went to church’.’
\citep[p.~690]{davidson:1963_orig}

The norm of \emph{agglomeration} says it is a mistake to knowingly have several intentions if it
would be a mistake to knowingly have one large intention agglomerating the several intentions; see
\citet{Bratman:1987xw} or \citet[§4]{setiya:2014_intention}.

‘why should rational agents like us have the capacity to have both ordinary intentions (subject to
demands for consistency and agglomeration) and guiding desires (which are not subject to these
demands)?’ \citep[pp.~137–8]{Bratman:1987xw}



\section{Motor Representation}

Markers of motor representation
\begin{enumerate}
\item are unaffected by variations in kinematic features but not goals
  \citep[e.g.][]{cattaneo:2010_state-dependent,umilta:2008pliers,cattaneo:2009_representation,rochat:2010_responses}
\item are affected by variations in goals but not kinematic features
  \citep[e.g.][]{Fogassi:2005nf,bonini:2010_ventral,cattaneo:2007_impairment,Umilta:2001zr,villiger:2010_activity,koch:2010_resonance}
\end{enumerate}
So:
\begin{enumerate}[resume]
\item carry information about goals (from 1,2)
\end{enumerate}
Also
\begin{enumerate}[resume]
\item Information about outcomes guides planning-like processes
  \citep[consider][]{grafton:2007_evidence,jeannerod:1998nbo,wolpert:1995internal, miall:1996_forward,arbib:1985_coordinated,mason:2001_hand,santello:2002_patterns}.
\end{enumerate}



\section{Motor Representations Ground the Directedness of Actions to Goals}



\section{Motor Representations Aren’t Intentions}

Imagining seeing an object move and actually seeing an object move have similarities in
characteristic performance profile
(\citealp{kosslyn:1978_measuring}; \citealp[p.\ 99ff]{kosslyn:1994_image}; \citealp{kosslyn:1978_visual})

The way imagining performing an action unfolds in time is
similar in some respects to the way actually performing an action of the same type would unfold
\citep{decety:1989_timing, decety:1996_imagined, Jeannerod:1994oz, parsons:1994_temporal,
frak:2001_orientation}

Judging the laterality of a hand vs of a letter.
For ordinary subjects, the tasks differ: they are less accurate
when the hand's position is biomechanically awkward.
But \citet{Fiori:2012fk} show that the tasks do not so differ for subjects suffering Amyotrophic
Lateral Sclerosis (ALS), which impairs motor representation \citep{parsons:1998_cerebrally}.

\begin{enumerate}
\item Only representations with a common format can be inferentially integrated.
\item Any two intentions can be inferentially integrated in practical reasoning.
\item My intention that I visit the ZiF is a propositional attitude.
\end{enumerate}
Therefore:
\begin{enumerate}[resume]
\item  All intentions are propositional attitudes
\end{enumerate}
But:
\begin{enumerate}[resume]
\item No motor representations are propositional attitudes.
\end{enumerate}
So:
\begin{enumerate}[resume]
\item No motor representations are intentions.
\end{enumerate}



\section{The Interface Problem}

The interface problem: explain how intentions and motor representations, with their distinct
representational formats, are related in such a way that, in at least some cases, the outcomes they
specify non-accidentally match.

Two collections of outcomes, A and B, \emph{match} in a particular context just if, in that context,
either the occurrence of the A-outcomes would normally constitute or cause, at least partially, the
occurrence of the B-outcomes or vice versa. To illustrate, one way of matching is for the B-outcomes
to be the A-outcomes. Another way of matching is for the B-outcomes to stand to the A-outcomes as
elements of a more detailed plan stand to those of a less detailed one.


Imagine that you are strapped to a spinning wheel facing near certain death as it plunges you into
freezing water. To your right you can see a lever and to your left there is a button. In deciding
that pulling the lever offers you a better chance of survival than pushing the button, you form an
intention to pull the lever, hoping that this will stop the wheel. If things go well, and if
intentions are not mere epiphenomena, this intention will result in your reaching for, grasping and
pulling the lever. These actions---reaching, grasping and pulling---may be directed to specific
outcomes in virtue of motor representations which guide their execution. It shouldn't be an accident
that, in your situation, you both intend to pull a lever and you end up with motor representations
of reaching for, grasping and pulling that very lever, so that the outcomes specified by your
intention match those specified by motor representations. If this match between outcomes variously
specified by intentions and by motor representations is not to be accidental, what could explain it?

‘As defined by Tutiya et al., an executable concept of a type of movement is a
representation, that could guide the formation of a volition, itself the proximal cause of
a corresponding movement. Possession of an executable concept of a type of movement thus
implies a capacity to form volitions that cause the production of movements that are
instances of that type.’
\citep[p.~7]{pacherie:2011_nonconceptual}

‘Motor schemas are thus, we submit, what bridges the gap between intentions and motor representations, ensuring proper, content-preserving coordination without requiring any mysterious translation process.’


    



%--- end paste
%---------------

\footnotesize
\bibliography{$HOME/endnote/phd_biblio}

\end{multicols*}

\end{document}
