%!TEX TS-program = xelatex
%!TEX encoding = UTF-8 Unicode

\documentclass[12pt]{extarticle}
% extarticle is like article but can handle 8pt, 9pt, 10pt, 11pt, 12pt, 14pt, 17pt, and 20pt text

\def \ititle {Origins of Mind}

\def \isubtitle {Lecture 01}

\def \iauthor {Stephen A. Butterfill}
\def \iemail{s.butterfill@warwick.ac.uk}
\date{}

%for strikethrough
\usepackage[normalem]{ulem}

\input{$HOME/Documents/submissions/preamble_steve_handout}

%\bibpunct{}{}{,}{s}{}{,}  %use superscript TICS style bib
%remove hanging indent for TICS style bib
%TODO doesnt work
\setlength{\bibhang}{0em}
%\setlength{\bibsep}{0.5em}


%itemize bullet should be dash
\renewcommand{\labelitemi}{$-$}

\begin{document}

\begin{multicols*}{3}

\setlength\footnotesep{1em}


\bibliographystyle{newapa} %apalike

%\maketitle
%\tableofcontents




%---------------
%--- start paste



\def \ititle {Lecture 06: Acting Together}

\begin{center}

{\Large

\textbf{\ititle}

}



\iemail %

\end{center}



\section{Contrast Cases and the Simple View}

\emph{Question}
What distinguishes genuine joint actions from parallel but merely individual actions?



\emph{Aim}
An account of joint action must draw a line between joint actions and parallel but
merely individual actions.



\emph{The Simple View}
:
Two or more agents perform an intentional joint action exactly when there is an act-type, φ, such that each of several agents intends that they, these agents, φ  together and their intentions are  appropriately related  to their actions.




\section{Walking Together in the Tarantino Sense}

`each agent does not just intend that the group perform the […] joint action. Rather, each agent intends as well that the group perform this joint action in accordance with subplans (of the intentions in favor of the joint action) that mesh' \citep[p.\ 332]{Bratman:1992mi}.

Our plans are \emph{interconnected} just if facts about your plans feature in mine and conversely.

‘shared intentional agency consists, at bottom, in interconnected planning agency of the participants’ \citep{Bratman:2011fk}.

\begin{minipage}{\columnwidth}

\emph{Bratman’s claim}. For you and I to have a collective/shared intention that we J it is sufficient that:

\begin{enumerate}[label=({\arabic*}),itemsep=0pt,topsep=0pt]

\item  `(a) I intend that we J and (b) you intend that we J;

\item `I intend that we J in accordance with and because of la, lb, and meshing subplans of la and lb; you intend that we J in accordance with and because of la, lb, and meshing subplans of la and lb;

\item `1 and 2 are common knowledge between us' \citep[View 4]{Bratman:1993je}

\end{enumerate}

\end{minipage}

\subsection{Shared Intention}

What distinguishes joint actions from parallel but individual actions?
‘the key property of joint action lies in [...] the participants’ having a [...] “shared” intention.’
\citep{alonso_shared_2009} %[pp.\ 444-5]

`I take a collective action to involve a collective intention.'  \citep[p.\ 5]{Gilbert:2006wr}

`The sine qua non of collaborative action is a joint goal [shared intention] and a joint commitment’
\citep[p.\ 181]{tomasello:2008origins}



\section{Multi-Agent Events}
Events $D_1$, ...\ $D_n$ \emph{ground} $E$, if: $D_1$, ...\ $D_n$ and $E$ occur;
$D_1$, ...\ $D_n$ are each (perhaps improper) parts of $E$; and
every event that is a proper part of $E$ but does not overlap  $D_1$, ...\ $D_n$ is caused by some or all of $D_1$, ...\ $D_n$.

For an individual to be \emph{among the agents of an event} is for there to be actions $a_1$, ...\ $a_n$ which ground this event where the individual is an agent of some (one or more) of these actions.

A joint action is an event with two or more agents.\citep{ludwig_collective_2007}



\section{Collective Goals}

An outcome is a \emph{collective goal} of two or more actions involving multiple
agents if it is an outcome to which those actions are collectively directed.



\section{Collective Goals and Motor Representations}

Motor representations concern not only bodily configurations and movements but also more distal
outcomes such as the grasping of a mug or the pressing of a switch
\citep{butterfill:2012_intention,hamilton:2008_action,cattaneo:2009_representation}.

Some motor processes are planning-like in that they involve deriving means by which the outcomes could
be brought about and in that they involve coordinating subplans
\citep{jeannerod_motor_2006,zhang:2007_planning}.

Motor processes concerning actions others will perform occur in observing others act
\citep{Gangitano:2001ft}---and even in observing several others act jointly
\citep{manera:2013_time}---and enables us to anticipate their actions
\citep{ambrosini:2011_grasping,aglioti_action_2008}.

In joint action, motor processes concerning actions another will perform can occur
\citep{kourtis:2012_predictive, meyer:2011_joint}, and can inform planning for one's own actions
\citep{vesper:2012_jumping,novembre:2013_motor,loehr:2011_temporal}.

In some joint actions, the agents have a single representation of the whole action (not only separate
representations of each agent's part)
\citep{tsai:2011_groop_effect,loehr:2013_monitoring,Menoret:2013fk}, and may each make a plan for both
their actions \citep{meyer:2013_higher-order,kourtis:2014_attention}.

An interagential structure of motor representation:
\begin{enumerate} \item there is an outcome to
which a joint action could be collectively directed and in each agent there is a motor representation
of this outcome; \item these motor representations trigger planning-like processes in each agent which
result in plan-like hierarchies of motor representations; \item the plan-like hierarchy in each agent
involves motor representations concerning another's actions as well as her own; \item the plan-like
hierarchies of motor representations in the agents nonaccidentally match.
\end{enumerate}






%--- end paste
%---------------

\footnotesize
\bibliography{$HOME/endnote/phd_biblio}

\end{multicols*}

\end{document}
