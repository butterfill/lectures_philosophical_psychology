%!TEX TS-program = xelatex
%!TEX encoding = UTF-8 Unicode

\documentclass[12pt]{extarticle}
% extarticle is like article but can handle 8pt, 9pt, 10pt, 11pt, 12pt, 14pt, 17pt, and 20pt text

\def \ititle {Origins of Mind}

\def \isubtitle {Lecture 01}

\def \iauthor {Stephen A. Butterfill}
\def \iemail{s.butterfill@warwick.ac.uk}
\date{}

%for strikethrough
\usepackage[normalem]{ulem}

\input{$HOME/Documents/submissions/preamble_steve_handout}

%\bibpunct{}{}{,}{s}{}{,}  %use superscript TICS style bib
%remove hanging indent for TICS style bib
%TODO doesnt work
\setlength{\bibhang}{0em}
%\setlength{\bibsep}{0.5em}


%itemize bullet should be dash
\renewcommand{\labelitemi}{$-$}

\begin{document}

\begin{multicols*}{3}

\setlength\footnotesep{1em}


\bibliographystyle{newapa} %apalike

%\maketitle
%\tableofcontents




%---------------
%--- start paste






\def \ititle {Lecture 01: Seeing Red: Do Humans Visually Experience Categorical Colour Properties?}

\begin{center}

{\Large

\textbf{\ititle}

}



\iemail %

\end{center}



\section{A ‘subject-determining platitude’ about colour}

‘If someone with normal color vision looks at a tomato in good light, the tomato will appear to have a distinctive property—a property that strawberries and cherries also appear to have, and which we call ‘red’ in English’ \citep[p.\ 4]{byrne:2003_color}

It is a ‘subject-determining platitude’
that ‘“red” denotes the property of an object putatively presented in visual experience
when that object looks red’, and likewise for other colour terms
\citep[pp.\ 199--200]{Jackson:1996zz}.

Simplifying assumption:



There is a property denoted by ‘red’ which some objects have;  call this property   red.


Premise 1:



If the property red (say) is presented in visual experience, then things which have this property  thereby differ in  visual appearance from things which do not have it.


Question:



Do red things differ in visual appearance from non-red things?


Objectives for this lecture:

understand questions about shared agency


can use the method of contrast cases


understand distributive and collective interpretations of sentences


can distinguish acting together from joint action


familar with the Simple View


can critically assess objections to the Simple View





%--- end paste
%---------------

\footnotesize
\bibliography{$HOME/endnote/phd_biblio}

\end{multicols*}

\end{document}
