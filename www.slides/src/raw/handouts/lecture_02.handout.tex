%!TEX TS-program = xelatex
%!TEX encoding = UTF-8 Unicode


\documentclass[12pt]{extarticle}
% extarticle is like article but can handle 8pt, 9pt, 10pt, 11pt, 12pt, 14pt, 17pt, and 20pt text

\def \ititle {Origins of Mind}

\def \isubtitle {Lecture 01}

\def \iauthor {Stephen A. Butterfill}
\def \iemail{s.butterfill@warwick.ac.uk}
\date{}

%for strikethrough
\usepackage[normalem]{ulem}

\input{$HOME/Documents/submissions/preamble_steve_handout}

%\bibpunct{}{}{,}{s}{}{,}  %use superscript TICS style bib
%remove hanging indent for TICS style bib
%TODO doesnt work
\setlength{\bibhang}{0em}
%\setlength{\bibsep}{0.5em}


%itemize bullet should be dash
\renewcommand{\labelitemi}{$-$}

\begin{document}

\begin{multicols*}{3}

\setlength\footnotesep{1em}


\bibliographystyle{newapa} %apalike

%\maketitle
%\tableofcontents




%---------------
%--- start paste



\def \ititle {Lecture 02: Pure Goal Tracking - A Developmental Puzzle}

\begin{center}

{\Large

\textbf{\ititle}

}



\iemail %

\end{center}

 
 
\section{When can infants first track goals?}
 
‘Six-month-olds and 9-month-olds showed a stronger novelty response (i.e., looked longer) on new-goal trials than on new-path trials (Woodward 1998). That is, like toddlers, young infants selectively attended to and remembered the features of the event that were relevant to the actor’s goal.'
\citep[p.\ 153]{woodward:2001_making}
 
 
 
\section{How are infants first able to track goals?}
 
\emph{The Teleological Stance}: The goals of an action are those outcomes which the means is a best available way of bringing about \citep{Gergely:1995sq,Csibra:1998cx}.

‘an action can be explained by a goal state if, and only if, it is seen as  the  most justifiable action towards that goal state that is available within the constraints of reality’
\citep[p.~255]{Csibra:1998cx}.



‘when taking the teleological stance one-year-olds apply the same
inferential principle of rational action that drives everyday mentalistic
reasoning about intentional actions in adults’ \citep{Csibra:2003kp}.
 
 
 \emph{The Simple View}
The principles comprising the Teleological Stance are things infants know or believe, and infants are
able to track goals by making inferences from these principles plus their beliefs about the means
agents have selected.
 


\section{A signature limit of infant goal tracking?}
 
\citet{Flanagan:2003lm} showed that
‘patterns of eye–hand coordination are similar when performing and observing a block stacking task’.
 
From at least three months of age, some of infants’ abilities to identify
the goals of actions they observe are linked to their abilities to perform
actions \citep{woodward:2009_infants}.
 
In adults, tying the hands impairs proactive gaze \citep{ambrosini:2012_tie}; in
infants, boosting grasping with ‘sticky mittens’ facilitates proactive gaze
(\citealp{sommerville:2005_action}; see also \citealp{sommerville:2008_experience},
\citealp{ambrosini:2013_looking, skerry:2013_firstperson}).
 
 
 
\section{The Motor Theory of Goal Tracking}
 

According to the  Motor Theory, infants’ (and adults’) pure goal-tracking sometimes depends on the double life of motor processes  \citep[see][for details]{sinigaglia:2015_puzzle}.
 
More carefully the \emph{Motor Theory of Goal Tracking} states that:
\begin{enumerate}
\item in action observation, possible outcomes of observed actions are represented motorically;
\item these representations trigger motor processes much as if the observer were performing actions directed to the outcomes;
\item such processes generates predictions;
\item a triggering representation is weakened if the predictions it generates fail.
\end{enumerate}
The result is that, often enough, the only only outcomes to which the observed action is a means
are represented strongly.
 
 
 
\section{A Developmental Puzzle about Goal-Tracking}
 
‘by the end of the first year infants are indeed capable of taking the intentional stance (Dennett, 1987) in interpreting the goal- directed behavior of rational agents.’
\citep[p.\ 184]{Gergely:1995sq}
 
‘12-month-old babies could identify the agent’s goal and analyze its actions causally in relation to it’
\citep[p.\ 190]{Gergely:1995sq}
 
 
 
\section{Perceptual Animacy}
 
Perceptual animacy is
the detection by broadly perceptual processes of animate objects and their targets
\citet[e.g.][]{gao:2009_psychophysics}.

In adults, abilities to perceptually detect chasing
depend on several cues including whether the chaser ‘faces’ its target (‘directionality’) and how
directly the chaser approaches its target (‘subtlety’).

The detection of animacy appears to be a broadly perceptual phenomena since it depends on areas of
the brain associated with vision and influences how perceptual attention is allocated
\citep{scholl:2013_perceiving} irrespective of your beliefs and intentions
\citep{buren:2016_automaticity}.

\vfill


%--- end paste
%---------------

\footnotesize
\bibliography{$HOME/endnote/phd_biblio}

\end{multicols*}

\end{document}
