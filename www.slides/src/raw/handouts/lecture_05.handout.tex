%!TEX TS-program = xelatex
%!TEX encoding = UTF-8 Unicode

\documentclass[12pt]{extarticle}
% extarticle is like article but can handle 8pt, 9pt, 10pt, 11pt, 12pt, 14pt, 17pt, and 20pt text

\def \ititle {Origins of Mind}

\def \isubtitle {Lecture 01}

\def \iauthor {Stephen A. Butterfill}
\def \iemail{s.butterfill@warwick.ac.uk}
\date{}

%for strikethrough
\usepackage[normalem]{ulem}

\input{$HOME/Documents/submissions/preamble_steve_handout}

%\bibpunct{}{}{,}{s}{}{,}  %use superscript TICS style bib
%remove hanging indent for TICS style bib
%TODO doesnt work
\setlength{\bibhang}{0em}
%\setlength{\bibsep}{0.5em}


%itemize bullet should be dash
\renewcommand{\labelitemi}{$-$}

\begin{document}

\begin{multicols*}{3}

\setlength\footnotesep{1em}


\bibliographystyle{newapa} %apalike

%\maketitle
%\tableofcontents




%---------------
%--- start paste




\def \ititle {Lecture 05: Do Humans Perceive Others’ Feelings}

\begin{center}

{\Large

\textbf{\ititle}

}



\iemail %

\end{center}

\citet[p.\ 573]{mcneill:2012_embodiment}: ‘We sometimes {see} aspects of each others’ mental lives, and thereby come to have non-inferential knowledge of them.’



\section{Categorical Perception \& Emotion}

There is a way of categorising static pictures of faces and other stimuli according to which emotion
someone might think they are expressing: some faces are happy, others fearful, and so on
From five months of age,
or possibly much earlier \citep{field:1982_discrimination},
through to adulthood, humans are better at distinguishing faces when they
differ with respect to these categories than when they do not
\citep{Etcoff:1992zd,Gelder:1997bf,Bornstein:2003vq,Kotsoni:2001ph,cheal:2011_categorical,hoonhorst:2011_categoricala}.

The patterns of discrimination do not appear to be an artefact of linguistic labels
(\citealp{sauter:2011_categorical}; see also \citealp{laukka:2005_categorical}, p.\ 291),%
nor of the particular choices subjects in these experiments are presented with \citep{bimler:2001_categorical,fujimura:2011_categorical}.
Nor are the patterns of discrimination due to narrowly visual features of the stimuli used \citep{sato:2009_detection}.

‘at a mean latency of 140 ms) the N170 showed both amplitude and latency
modulation differentially with emotional expressions. ...
[BUT] Whether this is due presently to low-level stimulus factors or to the use of emotional faces
is still to be determined’ \citep[p.~616]{batty:2003_early}.



\section{The Objects of Categorical Perception}

The same facial configuration can express intense joy or intense anguish depending on the posture of
the body it is attached to; and humans cannot accurately determine emotions from
spontaneously occurring (as opposed to acted out) facial
configurations \citep{motley:1988_facial,aviezer:2008_angry,aviezer:2012_body}.

Aviezer et al's puzzle:
Given that  facial configurations are not diagnostic of emotion, why  are they categorised by perceptual processes?

‘emotions are episodic modes of evaluative engagement with the social and practical world’
\citep[p.\ 1512]{parkinson:2008_emotions}.


 


%--- end paste
%---------------

\footnotesize
\bibliography{$HOME/endnote/phd_biblio}

\end{multicols*}

\end{document}
