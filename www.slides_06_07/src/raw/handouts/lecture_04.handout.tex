%!TEX TS-program = xelatex
%!TEX encoding = UTF-8 Unicode

\documentclass[12pt]{extarticle}
% extarticle is like article but can handle 8pt, 9pt, 10pt, 11pt, 12pt, 14pt, 17pt, and 20pt text

\def \ititle {Origins of Mind}

\def \isubtitle {Lecture 01}

\def \iauthor {Stephen A. Butterfill}
\def \iemail{s.butterfill@warwick.ac.uk}
\date{}

%for strikethrough
\usepackage[normalem]{ulem}

\input{$HOME/Documents/submissions/preamble_steve_handout}

%\bibpunct{}{}{,}{s}{}{,}  %use superscript TICS style bib
%remove hanging indent for TICS style bib
%TODO doesnt work
\setlength{\bibhang}{0em}
%\setlength{\bibsep}{0.5em}


%itemize bullet should be dash
\renewcommand{\labelitemi}{$-$}

\begin{document}

\begin{multicols*}{3}

\setlength\footnotesep{1em}


\bibliographystyle{newapa} %apalike

%\maketitle
%\tableofcontents




%---------------
%--- start paste




\def \ititle {Lecture 04: Seeing Causal Interactions}

\begin{center}

{\Large

\textbf{\ititle}

}



\iemail %

\end{center}

‘When we consider these objects with the utmost attention, we find only
that one body approaches the other; and that the motion of it precedes
that of the other, but without any sensible interval.’ \citep[p.~77]{Hume:1739lj}

‘the account flies in the face of our common-sense conviction that we do
perceive causal relations all the time. The experience of perceiving one
event following another is really quite different from the experience of
perceiving the second event as caused by the first’

‘the researches of Michotte and Piaget would seem to support our
common-sense view’
\citep[pp.~114-5]{Searle:1983tx}

Sometimes ‘a causal impression arises, clear,
genuine, and unmistakable, and the idea of cause can be derived from it
… in just the same way as the idea of shape or movement can be derived
from the perception of shape or movement.’
\citep[p.~270--1]{Michotte:1946nz}

‘the causal perception is the perception of the work of a mechanical force,
just as the impression of the movement of a car is the perception of its
displacement in physical space’
\citep[p.~228]{Michotte:1946nz}

‘This causal impression, however, would have been for him [Hume] ...
nothing but an illusion of the senses, as is shown by his
views with regard to the feeling of effort. ... [I]t is probable that his
[Hume’s] philosophical position would not have been affected in the
least.’
\citep[p.~256]{Michotte:1946nz}

‘In a great boulder rolling down the mountainside and flattening the wooden hut in its path we see an exemplary instance of force … these mechanical transactions … are directly observable (or experienceable)’
\citep[p.~118]{Strawson:1992yh}

‘just as the visual system works to recover … physical structure … by inferring properties such as 3-D shape, so too does it work to recover … causal … structure … by inferring properties such as causality’
\citep[p.~299]{Scholl:2000eq}

‘we seem to be as far as ever from deciding whether the hypothesis is
true: whether we perceive launchings rather than recognizing them by
means of stored patterns in long-term memory.’
\citep[p.~92]{rips:2011_causation}





\section{A Puzzle about the Development of Causal Understanding}

‘A similar permanent dissociation in understanding object support relations
            might exist in chimpanzees. They identify impossible support relations in looking tasks,
            but fail to do so in active problem solving.’
\citep{gomez:2005_species}

‘to date, adult primates’ failures on search tasks appear to
            exactly mirror the cases in which human toddlers perform poorly.’
\citep[p.\ 17]{santos:2009_object}



\section{Perception of Causation: Key Findings}

‘There are some cases … in which a causal impression arises, clear, genuine, and unmistakable,
and the idea of cause can be derived from it by simple abstraction in just the same way as the
idea of shape or movement can be derived from the perception of shape or movement’
\citep[p.\ 270--1]{Michotte:1946nz}

Infants at around six months of age seem also to distinguish launching from other sequences,
much as adults do \citep{Leslie:1987nr}.

‘… why it is that in our experiments certain particular conditions were
found necessary in order to give rise to a causal impression. They
correspond to the different characteristics of reproduction. …
anyone not very familiar with the procedure involved in framing the
physical concepts of inertia, energy, conservation of energy, etc., might
think that these concepts are simply derived from the data of immediate
experience’
\citep{Michotte:1946nz}



\section{Object Indexes}

‘the movement performed by object B appears simultaneously under two
different guises: (i) as a movement (belonging to object A), (ii) as a
change in relative position (by object B)’
\citep[p.~136]{Michotte:1946nz}

‘the physical movement of the object struck gives rise to a double
representation. This movement appears at one and the same time (a) as a
continuation of the previous movement of the motor object, and (b) as a
change of relative position (a purely spatial withdrawal) of the projectile
in relation to the motor object.’
\citep[p.~140]{Michotte:1946nz}

The \emph{object-specific preview benefit}: ‘observers can identify target
letters that matched the preview letter from the same object faster than
they can identify target letters that matched the preview letter from the
other object’ \citep[p.\ 2]{Krushke:1996ge}.





\section{Object Indexes and the Launching Effect}

\emph{Causal Object Index Conjecture}:
Effects associated with the ‘perception of causation’
are consequences of errors (or error-like patterns) in the assignments of object indexes and
their phenomenal effects.

Predictions:
(i) Where there is perception of causation, there will be errors (or error-like patterns)
in the assignments of object indexes.
(ii) Factors that can influence how object indexes are assigned or maintained can influence
perception of causation.

Objection: adaptation \citep{rolfs:2013_visual}.
But see further \citet{johnston:2013_causality,arnold:2015_objectcentered}.

‘Michotte and his followers worked out many of the factors which mediate
the perception of causality, such as the role of absolute and relative
speeds, spatial and temporal gaps in the objects' trajectories, differences
in the durations and angles of each object's trajectory, etc ...
‘This research has generally shown that many different  spatiotemporal parameters are critical for perceiving causality, but  that featural parameters (eg colors, shapes, sizes) play little or no role.’
\citep[p.~456]{Scholl:2004dx}

‘when there is a launching event beneath the overlap (or underlap event) timed such that
the launch occurs at the point of maximum overlap, observers inaccurately report that
the overlap is incomplete, suggesting that they see an illusory crescent.’
\citep[p.\ 461]{Scholl:2004dx}

Why does the illusory causal crescent appear?  Scholl and Nakayama suggest a
‘a simple categorical explanation for the Causal Crescents illusion: the visual system,
when led by other means to perceive an event as a causal collision, effectively
‘refuses’ to see the two objects as fully overlapped, because of an internalized
constraint to the effect that such a spatial arrangement is not physically possible.
As a result, a thin crescent of one object remains uncovered by the other one-as
would in fact be the case in a straight-on billiard-ball collision where the motion
occurs at an angle close to the line of sight.’
\citep[p.\ 466]{Scholl:2004dx}

‘object perception reflects basic constraints on the motions of physical bodies …’
\citep[p.\ 51]{Spelke:1990jn}

‘A single system of  knowledge  … appears to underlie object perception and physical reasoning’
\citep[p.\ 175]{Carey:1994bh}



\section{Phenomenal Expectations}


... are aspects of the overall phenomenal character of experiences which their subjects take to be informative about things that are only distantly related (if at all) to the things that those experiences intentionally relate the subject to.

Phenomenal expectations can be thought of as sensations in approximately Reid’s sense: they are monadic properties of events, specifically perceptual experiences,
which are individuated by their normal causes
and which alter the overall phenomenal character of those experiences in ways not determined by the experiences’ contents
(so two perceptual experiences can have the same content but distinct sensational properties).

Phenomenal expectations are signs:
they can lead to beliefs only via associations or further beliefs
(\citealp[Essay~II, Chap.~16, p.~228]{Reid:1785cj};
\citealp[Chap.~VI sect.~III, pp.~164–5]{Reid:1785nz}).




%--- end paste
%---------------

\footnotesize
\bibliography{$HOME/endnote/phd_biblio}

\end{multicols*}

\end{document}
